\documentclass{amsart}
\usepackage{amsmath,amsthm,amssymb}
%include these lines if you want to use the LaTeX "theorem" environments
\newtheorem{theorem}{Theorem}[section]
\newtheorem{definition}[theorem]{Definition}
\newtheorem{lemma}[theorem]{Lemma}
\newtheorem{corollary}[theorem]{Corollary}
\newtheorem{example}[theorem]{Example}
\newtheorem{proposition}[theorem]{Proposition}
%include lines like this if you want to define your own commands 
%to save typing
\newcommand{\PROOF}{\noindent {\bf Proof}: }
\newcommand{\REF}[1]{[\ref{#1}]}
\newcommand{\Ref}[1]{(\ref{#1})}
\newcommand{\dt}{\mbox{\rm   dt}}
\newcommand{\C}{\mathbb{C}}
\newcommand{\full}{\{1, \ldots, n\}}
\newcommand{\phat}{\hat{p}}
\begin{document}
\section{Introduction}\label{S:intro}

  Our main object of study is an algebraic complex $B(n, k)$ that is a filtration of the order complex of the 
Boolean algebra. More precisely, fix positive integers $k$ and $n$. Then define 
  $\Delta_l(n, k) = \{\varnothing \subset C_0 \subset \cdots \subset C_l \subset \{1, \ldots, n\}
  \colon 0 < |C_{i+1} - C_i| \le k 
  \mbox{ for all } -1 \le i \le l \}$. Here $C_{-1} = \varnothing$ and $C_{l+1} = \full$. Let $B_l(n, k)$ be the free abelian group generated by $\Delta_l(n, k)$ over $\C$.
We can now define our main object of study: 
\begin{definition}\label{:B(n,k)} Fix positive integers $n$ and $k$. Then define
  \begin{equation}
    B(n, k) = \bigoplus_{l=0}^{n-2} B_l(n, k).
  \end{equation}
\end{definition}
\begin{example} When $k > n-2$ we get the order complex of the Boolean algebra. When $k=1$ there are only maximal chains.
\end{example}
We must make this into a complex by defining a boundary operator. Suppose $C = \varnothing \subset C_0 \subset  \cdots
\subset C_l \subset \{1, \ldots, n\}$ is a chain in $\Delta_l(n, k)$.
Then define
\begin{equation}\label{E:delta}
  \partial_l(C) = 
  \begin{cases}
     \sum_{i=0}^l (-1)^i \varnothing \subset \cdots \subset {\hat C_i} \subset \cdots  \subset \full, 
	 &\text{if $|C_{i+1} - C_{i-1}| \le k$;}\\
     0, &\text{otherwise.}
  \end{cases}
\end{equation} where ${\hat C_i}$ denotes that this set is omitted from the chain. 
Extend $\partial_l$ linearly to a function on $B_l(n, k)$. Then it is easy to verify that $(B(n, k), \partial)$
is an algebraic complex.  



In the case of the Boolean algebra recall that
the natural action of $S_n$ on the poset $B_n$ defines an action of $S_n$ on the homology. Since this action doesn't 
effect step size of chains we also have an action of $S_n$ on $B(n, k)$, hence on $H_*(B(n, k))$ (we will always consider
complex coefficients when we take homology, so we omit the coefficients from the notation). We can now describe the
main motivation for considering this complex. If we just consider the trivial representation inside of $B(n, k)$ then we
note that since all numbers are equivalent we only need to concern ourselves with the sizes of the various sets. 
Introducing a formal variable $t$ as a place marker we see that the complex is the free abelian group over $\C$ generated
by $t^{i_1}\otimes \cdots \otimes t^{i_l}$, where $0 <i_j \le k$ and the exponents sum to $n$.
 Here the exponents keep track of how the chains grow. Examining the action of the boundary map we find that
\begin{equation} \label{E:Hochreduction}
  H_d(B(n, k))^{S_n} = HH_{d, n}(t\C[t]/(t^{k+1}))
\end{equation}
where $V^G$  denotes the trivial isotypic component of a representation $V$ of $G$ and $HH$ denotes the Hochschild 
homology (see Weibel \cite{Weibel} for information on Hochschild homology). Here the second 
subscript of $HH$ refers to the grading by degree. In fact the right-hand side of ~\ref{E:Hochreduction} is known, see 
Hanlon \cite{HanlonMac}:
\begin{equation} \label{E:HH}
  HH_{d, n}(t\C[t]/(t^{k+1})) = 
  \begin{cases} 
   1, &\text{if $d$ is odd and $\frac{(d+1)(k+1)}{2} = n;$}\\
   1, &\text{if $d$ is even and $\frac{d(k+1)}{2} = n;$}\\
   0, &\text{otherwise.}
  \end{cases}
\end{equation}

Gerstenhaber and Schack \cite{GS} defined a notion of Hodge decomposition for Hoch-schild homology and Hanlon \cite{Hanlon}
was able to extend this notion to posets that admit a Hodge structure. A Hodge structure on a poset is an action of
$S_l$ on chains of length $l$ which satisfies certain 
relations. Hanlon shows that the following actions define a Hodge structure on $B_n$:
\begin{definition}
  Let $C =\varnothing \subset C_0 \subset \cdots \subset C_l \subset \full$ be a chain of subsets. Fix $1 \le i \le l$. 
  Let $A = C_{i+1} - C_i$. Then for the transposition $\tau = 
  (i, i+1) \in S_l$ define $$\tau C = \varnothing \subset \cdots C_{i-1} \subset C_{i-1} \cup A \subset C_{i+1} \cdots 
  \subset \full.$$
\end{definition}

Note that this action will also work on $B(n, k)$. Now that we have described what the Hodge structure of $B(n, k)$ is (even though this is not poset
homology) we need to talk about the Hodge decomposition. Gerstenhaber and Schack defined pairwise orthgonal idempotents in
$\C S_n$, $e_n^{(1)}, \ldots, e_n^{(n)}$ called the Eulerian idempotents \cite{GS}. Unfortunately their definition is not
condusive to computation, see Loday \cite{Loday} for a more concrete definition. Since we have a Hodge structure on 
$B(n, k)$ we can define $$B_l^{(j)}(n, k) = e_l^{(j)} \cdot B_l(n, k).$$ The results of Hanlon (which follow Gerstenhaber
and Schack) show that $$\partial_l \colon B_l^{(j)}(n, k) \rightarrow B_{l-1}^{(j)}(n, k).$$ Thus we have that 
$$H_l(B(n, k)) = \bigoplus_j H_l^{(j)}(B(n, k))$$ and this is the {\bf Hodge decomposition of $H_*(B(n, k))$}. Further the
action of $S_n$ on $B(n, k)$ clearly commutes with the action of $S_l$ on $(l-1)$-chains. Thus each $H^{(j)}(B(n, k))$ is
an $S_n$ representation. Hence we have
\begin{equation}
  H_d^{(j)}(B(n, k))^{S_n} = HH_{d, n}^{(j)}(t\C[t]/(t^{k+1}).
\end{equation}

Unfortunately the complex is not a homology sphere. Thus by employing the theory in Hanlon 
\cite{Hanlon} we are only able to get results on the Euler characteristic of each Hodge piece. In particular we have:

\begin{theorem} \label{T:Hodgeintro} Let $\chi_j^{n, k}$ denote the euler characteristic of the $j$ Hodge piece of $B(n, k)$. 
Then we have $$\sum_n (-1)^n \sum_j \lambda^j Z(\chi_j^{n, k}) = - \prod_l (1 + a_l[Z(\epsilon_1) + \cdots + 
Z(\epsilon_k)])^{-\frac{1}{l}\sum{d|l} \mu(d) \lambda^{\frac{l}{d}}}.$$
\end{theorem} 

Here if $f$ is a class function of $S_n$, $Z(f)$ is the cycle indicator, that is $$Z(f) = \frac{1}{n!} \sum_{\sigma \in S_n}
 f(\sigma) Z(\sigma)$$ where $j_i(\sigma)$ denotes the number of $i$ cycles in $\sigma$ and $$Z(\sigma) = 
a_1^{j_1(\sigma)} a_2^{j_2(\sigma)} \cdots a_p^{j_p(\sigma)}.$$ Also the bracket operation is defined as $$A[B] = A[ a_i
\leftarrow B[a_j \leftarrow a_{ij}]]$$ where $\leftarrow$ denotes substitution. For example if $A = a_1^2 + a_2$ and 
$B= a_3+a_1$ then $A[B] = (a_3+a_1)^2 + a_6+a_2$. 




The rest of the paper is organized as follows: Section ~\ref{S:Hodge} proves Theorem ~\ref{T:Hodgeintro}, Section 
~\ref{S:oddend} gives various partial results 
and Section~\ref{S:data} gives some data we have generated.



\section{Hodge results}\label{S:Hodge}

Our goal is to derive an expression for the generating function of the Hodge pieces, that is to prove 
Theorem ~\ref{T:Hodgeintro}. 
   
 We do this by following section 2 of Hanlon \cite{Hanlon}. The idea is to use the following identity
(See proof of Theorem 2.1 in Hanlon \cite{Hanlon}):
\begin{equation} \label{E:OldHan}
  \sum_{j \ge 1} \sum_{p \ge 0} \sum_{\tau \in S_p} [e_p^{(j)}]_\tau Z(\tau) \lambda^j = \prod_l (1 + (-1)^l a_l)^
  {-\frac{1}{l} \sum_{d |l} \mu(d) \lambda^{\frac{l}{d}}}.
\end{equation}

In addition to this identity we will also need a result concerning the following object:
\begin{definition} Let $\tau \in S_{u+1}$. Then define $\omega^{(n)}_\tau$ to be the class function whose value on 
$\sigma \in S_n$ is the number of $u$-chains fixed in $B(n, k)$ by $(\tau, \sigma)$. 
\end{definition}

The result we need is:
\begin{lemma} \label{L:mainlemma} Fix $\tau \in S_{u+1}$. Then 
  \begin{equation} \sum_n Z(\omega^{(n)}_\tau) = Z(\tau)[Z(\epsilon_1) + \cdots + Z(\epsilon_k)]
  \end{equation}
\end{lemma}

\begin{proof} (See proof of lemma 2.2 in \cite{Hanlon}.)
  Let $T_1, \ldots, T_s$ be the cycles of $\tau$ with $|T_i| = t_i$ and $\sigma \in S_n$. 
  Let $C$ be a chain fixed by $(\tau, \sigma)$ and $A_i$
  be the subset added in $C$ at step $u_i$. For $C$ to be fixed we need $\sigma(A_i) = \sigma(A_{i-1})$ (here $A_0 = A_l$).
  Then $|A_i| = c$ for some number $c$ independent of $i$. Let $A = \cup_{i=1}^l A_i$, $\sigma_i = \sigma|_{A_i}$. Our 
  requirement on $\sigma_i$ is that  it is an injective function from $A_i$ to $A_{i-1}$ with $Z(\sigma|_A) = x_l[Z(
  \sigma_l \sigma_{l-1} \cdots \sigma_1)]$. In fact we get 
  \begin{equation} \label{E:lemma}
    \sum_{\sigma_1, \ldots, \sigma_l} Z(\sigma|A) = (c!)^{l-1} \sum_{\sigma \in S_c} a_l[Z(\sigma)] = 
    (c)^l a_l[Z(\epsilon_m)].
  \end{equation}
  So to calculate the number of chains fixed by $(\tau, \sigma)$, first pick for each $T_i$ a subset $S_i$ to play to role
  of $A$. There are ${n \choose m_1t_1, \ldots m_st_s}$  ways to do this. Then we need to divide each $S_i$ into equal 
  pieces to be added at each step of $T_i$. This can be done in ${m_it_i \choose m_i, \ldots, m_i}$ ways. Combining these
  results with Equation ~\ref{E:lemma} we get $$\sum_n Z(\omega^{(n)}_\tau) = \sum_n \frac{1}{n!} \sum_{\sigma \in S_n}
  |\{\mbox{chains fixed by }(\tau, \sigma)\}|Z(\sigma)$$ $$ = \sum_{k \ge m_j \ge 1} \frac{1}{n!} 
  {n \choose m_1t_1, \ldots, 
  m_st_s}\prod_i {m_it_i \choose m_i, \ldots m_i} (m_i)^{t_i} a_{t_i}[Z(\epsilon_{m_i})]$$ $$ = \sum_{k \ge m_j \ge 1} 
  \prod_{i=1}^s 
  a_{t_i}[Z(\epsilon_{m_i})] = \prod_{i=1}^s a_{t_i}[Z(\epsilon_1) + \cdots + Z(\epsilon_k)].$$
\end{proof}

Now that we have Lemma~\ref{L:mainlemma} we can proceed with proving Theorem~\ref{T:Hodgeintro}.

\begin{proof} (See proof of Theorem 2.1 in \cite{Hanlon})
  We wish to get an expression for $$\sum_n (-1)^n \sum_j \lambda^j Z(\chi_j^{n, k}).$$ We can rewrite this as: 
  \begin{equation} \label{E:def}
    \sum_j \lambda^j \sum_n \frac{1}{n!} \sum_{\sigma \in S_n} \chi_j^{n, k}(\sigma)Z(\sigma) (-1)^n. 
  \end{equation}
  If we denote the $jth$ Hodge piece of the chain complex by $C_*^{(j)}$ we have that
  $$\chi_j^{n, k} = \sum_{i=0}^{n-2} (-1)^i tr(\sigma|_{C_{r-i}^{(j)}}) = \sum_{i=0}^r (-1)^i tr(\sigma e_{r+1-i}^{(j)}
  |_{C_{r-i}})$$ $$= \sum_{i=0} (-1)^i \sum_{\tau \in S_{r+1-i}} [e^{(j)}]_\tau tr(\sigma\tau|_{C_{r-i}}).$$ Combining this 
  with Equation ~\ref{E:def} we get $$\sum_j \lambda^j \sum_n \frac{(-1)^n}{n!} \sum_{\sigma \in S_n}
  \sum_{i=0}^n (-1)^i \sum_{\tau \in S_{n+1-i}} [e^{(j)}_{n+1-i}]_\tau \omega_\tau^{(n)}(\sigma) Z(\sigma)$$
  $$ = \sum_j \lambda^j \sum_{p=1}^\infty (-1)^{p+1} \sum_{\tau \in S_p} [e_p^{(j)}]_\tau (\sum_n \frac{1}{n!} \sum_{\sigma
  \in S_n} \omega_\tau^{(n)}(\sigma)) Z(\sigma).$$ Then by applying Lemma~\ref{L:mainlemma} we get 
  $$=\sum_j \sum_{p=1}^\infty \frac{(-1)^p}{p!} \sum_{\tau \in S_p} [e_p^{(j)}]_\tau \lambda^j Z(\tau)[Z(\epsilon_1) + 
  \cdots + Z(\epsilon_k)].$$ Which by Equation~\ref{E:OldHan} becomes $$-\prod_l (1+ a_l[Z(\epsilon_1) + \cdots + 
  Z(\epsilon_k)])^{- \frac{1}{l} \sum_{d | l} \mu(d) \lambda^\frac{l}{d}}.$$
\end{proof}

  

 \begin{corollary} Let $\chi_{n, k}$ denote the Euler characteristic on $B(n, k)$. Fix $k$. Then 
   \begin{equation} \label{E:Euler}
        \sum_n (-1)^n Z(\chi_{n, k}) = -\frac{1}{1 + Z(\epsilon_1) + \cdots + Z(\epsilon_k)}.
   \end{equation}
\end{corollary}
\begin{proof}
   We need to evaluate the result from Theorem~\ref{T:Hodgeintro} with $\lambda = 1$. Using the well-known identity
   \begin{equation} \sum_{d|l} \mu(d) =   \begin{cases} 
   1, &\text{if $l=1$;}\\
   0, &\text{otherwise,}
  \end{cases}
  \end{equation} we see that we get $$\sum_n (-1)^n Z(\chi_{n, k}) =- \frac{1}{1+Z(\epsilon_1)+\cdots +Z(\epsilon_k)}$$ 
  since $a_1[A]=A$ for all $A$.
\end{proof}


We can derive a further result from Theorem~\ref{T:Hodgeintro}. 

\begin{proposition} \label{P:sign1} Fix $k$ and let $\eta_n$ be the sign representation of $S_n$. Then we have:
  \begin{equation}
    \sum_n (-1)^n(\sum_l <\eta_n, \chi_l^{n, k}> \lambda^l)x^n = -\frac{1}{1-x\lambda}
  \end{equation}
\end{proposition}
\begin{proof}
$$  \sum_n (-1)^n(\sum_l <\eta_n, \chi_l^{n, k}> \lambda^l) x^n = \sum_{n, l} (-1)^n \frac{\lambda^l}{n!} \sum_{\sigma \in S_n}
    \chi_l^{n, k}(\sigma) \eta_n(\sigma) x^n$$
$$ =\sum_{n, l} (-1)^n \frac{\lambda^l}{n!} \sum_{\sigma \in S_n} \chi_l^{n, k}(\sigma) (Z(\sigma)[a_l \leftarrow (-1)^{l+1}x^l])$$
$$ =\sum_n (-1)^n \sum_l \lambda^l (\sum_{\sigma \in S_n} \frac{1}{n!} \chi_l^{n, k}(\sigma) (Z(\sigma)[a_l \leftarrow (-1)^{l+1}x^l])$$
$$ =\sum_n (-1)^n \sum_l \lambda^l Z(\chi_l^{n, k})[a_l \leftarrow (-1)^{l+1}x^l]$$
\begin{equation} \label{E:sign2} =-\prod_l (1+a_l[Z(\epsilon_1) + \cdots + Z(\epsilon_k)])^{-\frac{1}{l} \sum_{d|l} \mu(d) \lambda^{\frac{l}{d}}}[a_l \leftarrow (-1)^{l+1}x^l].
\end{equation}
So the next step is to calculate $(a_l[Z(\epsilon_i)])[a_l \leftarrow (-1)^{l+1}x^l]$. Notice a term $a=a_1^{p_1} \cdots 
a_r^{p_r}$ is first sent to $a_l^{p_1} \cdots a_{lr}^{p_r}$ and then finally to
\begin{equation} \label{E:sign1}
  (-1^{l+1}x^l)^{p_1} \cdots (-1^{lr+1}x^{rl})^{p_n} =(-1)^{ln + \sum_{i=1}^r p_i}x^{nl}.
\end{equation} Recall that the sign of $a$ is $(-1)^{\sum_{i=1}^r (i+1)p_i} = (-1)^{n+\sum_{i=1}^r p_i}.$ Thus using this we get that
$$(a_l[Z(\epsilon_n)])[a_l \leftarrow (-1)^{l+1}x^l] =  x^{ln} \sum_{\sigma \in S_n} (-1)^{nl}(-1)^{-n} sgn(\sigma).$$ This
sum is proportional to the intertwining number of $\epsilon_n$ and $\eta_n$. Thus it is zero unless $n=1$. Hence returning
to Equation~\ref{E:sign2} we get
$$-\prod_l (1+ x^l)^{-\frac{1}{l} \sum_{d|l} \mu(d) \lambda^{\frac{l}{d}}}.$$ Applying 
Equation 6.2a in \cite{Hanlon} we get the above result. 
\end{proof}

Thus the sign representation appears only once and in the top Hodge piece. Latter we will give another way to derive this 
result. 
%For now we will push the above techinque further. 
%\begin{proposition} \label{P:trivial} Fix $k$. Then we have:
%  \begin{equation}
%    \sum_n (-1)^n(\sum_l <\epsilon_n, \chi_l^{n, k}> \lambda^l)x^n = -\frac{1+x\lambda}{1-x^k\lambda}
%  \end{equation}
%\end{proposition} 
%\begin{proof}
%  Proceeding as above we get 
%  $$\sum_n (-1)^n(\sum_l <\epsilon_n, \chi_l^{n, k}> \lambda^l)x^n = \sum_n (-1)^n \sum_l \lambda^l Z(\chi_l^{n, k})[a_l \leftarrow x^l]$$
%  \begin{equation} \label{E:triv} =-\prod_l (1+a_l[Z(\epsilon_1) + \cdots + Z(\epsilon_k)])^{-\frac{1}{l} \sum_{d|l} \mu(d) \lambda^{\frac{l}{d}}}[a_l \leftarrow x^l].
%  \end{equation}
%  Here though it is easier to simplify this. We get 
%  $$-\prod_l(1 + x^l + x^{2l} + \cdots + x^{kl})^{-\frac{1}{l} \sum_{d|l} \mu(d) \lambda^{\frac{l}{d}}}
%   =-\prod_l (\frac{1-x^{l(k+1)}}{1-x^l})^{-\frac{1}{l} \sum_{d|l} \mu(d) \lambda^{\frac{l}{d}}}.$$  
%\end{proof}
\section{Odds and Ends} \label{S:oddend}
In this section we mention various results and comments. The first remark is that we can identify a chain with a tabloid as
follows: row $i$ of the tabloid is $C_i - C_{i-1}$. Using this we can refine Proposition~\ref{P:sign1} again. 
\begin{proposition} \label{P:sign2} The sign representation occurs only in the top homology class and with multiplicity one. 
Further if $k > 1$ this is the only representation in the top homology class.
\end{proposition}
\begin{proof} 
  Write the chain complex $B(n, k) = \bigoplus_u C_u$ 
  where the sum is over all compositions of $n$ with maximum part $k$ and $C_u$ corresponds to chains of type $u$. $C_u$ 
  corresponds to tabloids of shape $u$. The sign 
  representation of $S_n$ corresponds to the shape $(1^n)$. The multiplicity of the sign representation in $C_u$ 
  corresponds to the number of semistandard Young tableaux of shape $(1^n)$ and type $u$ (see section 2.9 in 
  \cite{Sagan}). Thus this only occurs when $u = (1^n)$. Thus on $C_{(1^n)}$ the boundary operator is zero, so the sign
  representation survives in homology.   
 
  Note further that $k > 1$ then all chains in dimension $n-3$ are there. Hence in $H_{n-2}$ there are no cycles that are
  not in the Boolean algebra. Thus we only get the sign representation.
\end{proof}


We also have a second method of calculating the Euler characteristic:
 \begin{proof} 
    Recall (See section 2.11 in \cite{Sagan}) that the $S_n$-module of 
    tabloids of shape $\mu$ is isomorphic to $\bigoplus_{\lambda} K_{\lambda, \mu} S^\lambda$ where $K_{\lambda, \mu}$
    is the Kostka number and $S^\lambda$ is the Specht module of shape $\lambda$. Then the $$\sum_n \chi_{n, k} = 
    \sum_n (-1)^n \sum_{\mu| \mu_i \le k} \sum_\lambda K_{\lambda, \mu} S^\lambda = \sum_n (-1)^n \sum_\lambda 
    S^\lambda \sum_{\mu | \mu_i \le k} K_{\lambda, \mu}.$$ So the coefficient of $S^\lambda$ is the number of semistandard 
    Young tableaux of shape $\lambda$ and the multiplicity of any number is no bigger than $k$. Thus if we think about 
    building up such a tableaux we can add no more than $k$ blocks at once. Thus we get the above equation. 
  \end{proof}

So we can interpret the righthand side of Equation~\ref{E:Euler} as being the signed sum of all ways of adding at most $k$ 
blocks. This idea leads to the calculation of $H_*(B(n, n-2))$.

\begin{proposition} $H_d(B(n, n-2))$ is zero except in dimension $1$, where it is $(S^{n-1, 1})^2 \bigoplus S^n$.
\end{proposition}
\begin{proof}
  By the next result we know the homology is zero (except for the sign representation in the top dimension) above dimension
  $1$. Fix $n$ If we examine the righthand side of Equation~\ref{E:Euler} with $k=n-2$ then the only difference from the 
  Boolean algebra case is we cannot add a block of size $n-1$. In the case $k=n-1$ the righthand side evaluates to 
  $(n) + (-1)^{n-1} (1^n)$. Now if we could add a block of size $n-1$ we could only add it to a block of size 1. Hence we 
  get $(1)*(n-1) + (n-1)*(1) = 2(n) + 2(n-1, n)$ So we subtract these from the $k=n-1$ case to get that the Euler 
  characteristic of the $n-2$ case is $-(n) -2(n,n-1) + (-1)^{n-1}(1^n)$. Thus $(n)+2(n, n-1)$ must appear in dimension
  $1$. 
\end{proof}

Now we give our last result, using spectral sequences. 
\begin{proposition}[Upper-triangularity] \label{P:uppertri} Except for the sign representation as described in Proposition~\ref{P:sign1} the 
  homology vanishes above dimension $n-k-1$. 
\end{proposition}
\begin{proof}
  Let $F(\varnothing \subset C_1 \subset \cdots \subset C_d \subset \{1, \ldots, n\}) = |C_1|$. The boundary on the
  $E^1$ is the normal boundary map except we cannot remove the first set. Thus if we look at the $E^2$
  term of the induced spectral sequence it is easy to see that 
  \begin{equation} E_d^2(n, k) \cong \oplus_{A \subset \full \colon |A| \le k} H_{d-1}(n-|A|, k).
  \end{equation}
  From this the claim follows easily by induction.
\end{proof}



\section{Data} \label{S:data}
Note that for $k=1$ we get the regular representation in the top dimension and for $k=n-1$ we get the homology of a sphere.

  \begin{table}
\begin{center}
\begin{tabular}{ccccc}
k= & 1 & 2 & 3 & 4 \\
$H_0(B(3, k))$ & $0$ & $S^3$ &  &  \\
$H_1(B(3, k))$ & $\C S_3$ & $S^{111}$ &  &  \\
$H_0(B(4, k))$ & $0$ & $0$ & $S^4$ &  \\
$H_1(B(4, k))$ & $0$ & $S^{31} \oplus S^{31} \oplus S^4$ & $0$ &  \\
$H_2(B(4, k))$ & $\C S_4$ & $S^{1111}$ & $S^{1111}$ &  \\
$H_0(B(5, k))$ & $0$ & $0$ & $0$ & $S^5$ \\
$H_1(B(5, k))$ & $0$ & $0$ & $S^{41} \oplus S^{41} \oplus S^5$ & $0$ \\
$H_2(B(5, k))$ & $0$ & $S^{311} \oplus S^{311} \oplus S^{32} \oplus S^{41} \oplus S^{41}$ & $0$ & $0$ \\
$H_3(B(5, k))$ & $\C S_5$ & $S^{11111}$ & $S^{11111}$ & $S^{11111}$ \\
\end{tabular}
\caption[Homologies]{}
\label{Homologies}
\end{center}
\end{table}

Using fixed point theorems we can compute character values:
\begin{table}
\begin{center}
\begin{tabular}{cccccc}
$\lambda=$ & $1111$ & $211$ & $22$ & $31$ & $4$ \\
$d=0, k=1$ & 0 & 0 & 0 & 0 & 0 \\
$d=1 k=1$ & 0 & 0 & 0 & 0 & 0 \\
$d=2, k=1$ & 24 & 0 & 0 & 0 & 0 \\
$d=0, k=2$ & 0 & 0 & 0 & 0 & 0 \\
$d=1 k=2$ & 7 & 3 & -1 & 1 & -1 \\
$d=2 k=2$ & 1 & -1 & 1 & 1 & -1 \\
$d=0, k=3$ & 1 & -1 & 1 & 0 & 1 \\
$d=1, k=3$ & 0 & 0 & 0 & 0 & 0 \\
$d=2, k=3$ & 1 & -1 & 1 & 1 & -1 \\
\end{tabular}
\caption[Character Values]{Character Values for $n=4$}
\label{Char}
\end{center}
\end{table}



\begin{thebibliography}{9}
\bibitem{GS}
  M. Gerstenhaber and S.D. Schack,
  \emph{A Hodge-type decomposition for commutative algebra cohomology,}
  J. Pure Appl. Algebra {\bf 48} (1987), 229-247.
\bibitem{HanlonHodge}
  P. Hanlon, 
  \emph{The Action of $S_n$ on the Components of the Hodge Decomposition of Hoschschild Homology}
  Michigan Math. J. {\bf 37} (1990), 105-124.
\bibitem{HanlonMac}
  P. Hanlon,
  \emph{Cyclic homology and the Macdonald conjectures,}
  Invent. Math. {\bf 86} (1986), 131-159.
\bibitem{Hanlon}
  P. Hanlon,
  \emph{Hodge Structure on Posets}
\bibitem{Loday}
  J. L. Loday,
  \emph{Partition eul\'eriene et op\'erations en homologie cyclique.}
  C. R. Acad. Sci. Paris S\'er. I Math. {\bf 307} (1988), 283-286.
\bibitem{Sagan}
  B. E. Sagan,
  \emph{The Symmetric Group: Representations, Combinatorial Algorithms, and Symmetric Functions}
  Springer-Verlag, NY, 2001
\bibitem{Weibel}
  C. Weibel,
  \emph{An introduction to homological algebra,}
  Cambridge University Press, UK, 1994.

\end{thebibliography}
\end{document}

\documentclass{amsart}
\usepackage{amsmath,amsthm,amssymb}
%include these lines if you want to use the LaTeX "theorem" environments
\newtheorem{theorem}{Theorem}
\newtheorem{definition}[theorem]{Definition}
\newtheorem{lemma}[theorem]{Lemma}
\newtheorem{corollary}[theorem]{Corollary}
\newtheorem{example}[theorem]{Example}
\newtheorem{proposition}[theorem]{Proposition}
%include lines like this if you want to define your own commands 
%to save typing
\newcommand{\PROOF}{\noindent {\bf Proof}: }
\newcommand{\REF}[1]{[\ref{#1}]}
\newcommand{\Ref}[1]{(\ref{#1})}
\newcommand{\dt}{\mbox{\rm   dt}}
\newcommand{\C}{\mathbb{C}}
\newcommand{\full}{\{1, \ldots, n\}}
\newcommand{\phat}{\hat{p}}
\begin{document}

%\title{}

%\author{Scott Kravitz}
%\address{Department of Mathematics, University of Michigan, Ann Arbor, MI 48109-1003} 
%\subjclass{Primary 05E25}
\date{\today}     
%\begin{abstract}
%\end{abstract}
%\maketitle

\begin{proposition}[Upper-triangularity] \label{P:uppertri} Fix $n$ and $k$, and set $x=n-k$. 
  Except for the sign representation, which occurs with multiplicity 1 in
  dimension $n-2$, the homology of $B(n, k)$ vanishes above dimension $x-1$. 
\end{proposition}

\begin{theorem} Fix $n$ and $k > 1$ and set $x=n-k$. Unless $\lambda$ is of the form $(k+1, 2^y, 1^{x-1-2y})$ or 
$(k+2, 2^y, 1^{x-2-2y})$ the homology in dimension $x-1$ is zero. Further we have
  \begin{enumerate}
    \item dim($H_{x-1}( \Delta_k(k+1, 2^y, 1^{x-1-2y}) )) = x-2y$
    \item dim($H_{x-1}( \Delta_k(k+2, 2^y, 1^{x-2-2y}) )) = x-2y-1.$
  \end{enumerate}
\end{theorem}

\begin{proof}
  We begin by using a spectral sequence. If $t$ is a semi-standard Young tableau then let $F(t)$ denote the number of 
  maximal elements in $t$. Notice then that since the boundary operator always combines numbers, that after applying
  the boundary operator we always have the same or more maximal elements (though the actual maximal element may change).
  Hence we get:  $$F(\partial(t)) \ge F(t).$$  Thus $F$ gives a filtration on $\Delta_k(\lambda)$ for any lambda and we
  can consider the associated spectral sequence. 

  The next step is to compute the $E^1$ term of the spectral sequence, that is to compute the homology of the complex
  $(\Delta(\lambda), \partial_0)$ where $\partial_0$ is the part of the boundary operator that does not change the 
  filtration index. Since the filtration index only measures the number of maximal elements of $t$, $\partial_0$ can adjoin
  any two numbers execpt for the maximal and maximal minus 1. Thus the blocks of $t$ containing maximal elements are never
  changed in any way by $\partial_0$. Further if we removed these blocks we would still get a semi-standard Young tableau
  since we can always choose a maximal element that lies in an outside corner, and then do the same for the tableau we 
  have left. What we have removed when we do this is a horizontal strip. Then $\partial_0$ acts the same as $\partial$ on 
  what we have left. Hence we get:
  \begin{equation}\label{E:E1term}
    E^1(\Delta_k(\lambda)) \cong \bigoplus H_{x-2}(\Delta_k(\mu)).
  \end{equation}
  Where we sum over all $\mu \subset \lambda$ such that $\lambda - \mu$ is a horizontal strip with size less than $k$.

  Now we apply Proposition~\ref{P:uppertri} to the right hand side of Equation~\ref{E:E1term}. Letting $s=|\lambda - \mu|$
  we wish to know if 
  $$ x-2=n-k-2 >|\mu| -k -1=n-s-k-1.$$ We get that the solution is $$s > 1.$$ Unless, of course, we remove a horizontal 
  strip and get $1^n$, but then we started with a hook shape.

  Hence if we remove more than one block, the
  homology vanishes so we need only sum over $|\lambda - \mu| = 1$ in Equation~\ref{E:E1term} (except for the hook  
  shape case). 

  Now we set up the induction argument. We will induct on $n$, so assume the theorem is true for all $n < N$ 
  (letting $X=N-k$). Then for most
  shapes removing an outside corner in Equation~\ref{E:E1term} will result in a shape of size $N-1$ which is not on the
  list in the theorem. Thus the right-hand side is zero and we are done. Since we can only remove $k$ boxes, they only
  hook to get to the sign representation is a shape where the top row is of size $k+1$ or less. 
  The only shapes that we have to worry about then
  are: 
  \begin{enumerate}
    \item Any hook shape of size $N$ where the first row is not of size less than $k+1$.
    \item $(k+3, 2^y, 1^{X-3-2y})$
    \item $(k+1, 3, 2^y, 1^{X-4-2y})$
    \item $(k+2, 3, 2^y, 1^{X-5-2y})$
  \end{enumerate}

  So first we eliminate the hook case: $(a, 1^{N-a})$. 
  We start with a shape where the first row is less than $k+1$ or
  the first row is more than $k+2$. In the first case the partitions on the right hand side of Equation~\ref{E:E1term} 
   become: $1^{N-a}$ and $1^{N-a-1}$. We got to these shapes
  by removing $a$ [resp. $a+1$] $m$'s. There is only one generator of the sign homology and this is in the top dimension.
  Thus $H_{X-2}(1^y)$ is nonzero only when $X=y$. That will occur only if $X-2 = N-a$ or $X-2 = N-a-1$, that is only if 
  $a= k+1$ or $a=k+2$. To summarize for latter use we have:
  \begin{equation} \label{E:smallhook}
    H_y(a, 1^{n-a}) \cong 0 \mbox{ if $a < k+1$ and $y \ne n-2$}
  \end{equation}

  Now we can move on to the next cases. In all of these cases when we apply Equation~\ref{E:E1term} there is one 
  important commonality: there is a single term on the right hand side of the equation. This means that in succesive terms
  of the spectral sequence elements of the kernel of a $\partial_i$ can result from cancellation. That is because every 
  tableau has a single maximal element in the same place. Thus since $\partial_i$ will only combine that maximal element
  with elements that are one less, the effect of taking $\partial_i$ is the same as placing an maximal-1 element there 
  instead. So if no cancellation occured before, it won't occur now.

  So any cycles that survive must be caused either by (1) violated column strictness or (2) exceeding $k$. To argue against
  these we need to keep careful track of where the maximal elements are, so the following lemma is useful:

  \begin{lemma}\label{L:toprow}
    Fix $n$ and $k > 1$, set $x=n-k$. Suppose $t$ is a homology generator with maximal element $m$ in either 
    $H_{x-1}(\Delta_k(k+1, 1^{n-k-1}))$ or $H_{x-1}(\Delta_k(k+2, 1^{n-k-2}))$. Then the top row of $t$ begins with a $1$ 
    and then $k$ of some number followed by another number in the $k+2$ case. 
  \end{lemma}
  \begin{proof}
    We proceed by induction on $n$. So assume the lemma is true for $n, k < N$. The base case is easily verified by hand.
    Now consider Equation~\ref{E:E1term}. We first take the partition $(k+1, 1^{n-k-1})$. The equation becomes: 
    \begin{equation}\label{E:E1term-hookcase-1}
      H_{x-2}(\Delta_k(1^{n-k})) \oplus 
      H_{x-2}(\Delta_k(k, 1^{n-k-1})) \oplus H_{x-2}(\Delta_k(k+1, 1^{n-k-2})).
    \end{equation}

    Since we obtained this equation by removing all the $m$'s from the tableau, we see that in the case of the  sign 
    representation we are done, because there $k$ $m$'s appear in the first row (note that we don't have $1^{n-k-1}$ since
    in the that  sign case we've 
    taken away $k+1$ $m$'s, which can't happen, so that term is zero). In the third term of the equation the induction 
    hypothesis applies. That just leaves the $(k, 1^{n-k-1})$ term. However by 
    Equation~\ref{E:smallhook} the homology of this is zero since $x-2 = n-k -2 < n-3 = $ the size of $(k, 1^{n-k-1})$
    ($k>1$ by hypothesis).

    Next we examine the $(k+2, 1^{n-k-2})$ case. Here the equation becomes: 
    \begin{equation}\label{E:E1term-hookcase-2} 
      H_{x-2}(\Delta_k(k+1, 1^{n-k-2})) \oplus H_{x-2}(\Delta_k(k+2, 1^{n-k-3})).
    \end{equation}
     
    In both these cases the induction hypothesis applies, so we are done.
  \end{proof} 

  First we will deal with the possibility that we violate column strictness. This can only happen with case 3 and 4. Here
  there is, after applying the appropriate $\partial_i$,  a $m-1$ in the third box of the second row and in the third box 
  of the first row there also a $m-1$. Thus by Lemma~\ref{L:toprow} there is a $m-1$ in the second box of the first row, 
  hence since the second box of the first row was not a $m$ and thus not changed by $\partial_i$, there is nothing that can
  go there without violating column strictness of the second column. So this case cannot happen.

  That just leaves us with the case of exceeding $k$. In cases 2 and 4 this cannot happen because by Lemma~ref{L:toprow} 
  the only repeated number is not the maximal in the $k+2$ case. In case 3 we can have the maximal number be repeated,
  but then this will reduce to the case of violating column strictness, which was already covered.
\end{proof}

\end{document}

\documentclass{amsart}
\usepackage{amsmath,amsthm,amssymb}
%include these lines if you want to use the LaTeX "theorem" environments
\newtheorem{theorem}{Theorem}[section]
\newtheorem{definition}[theorem]{Definition}
\newtheorem{lemma}[theorem]{Lemma}
\newtheorem{corollary}[theorem]{Corollary}
\newtheorem{example}[theorem]{Example}
%include lines like this if you want to define your own commands 
%to save typing
\newcommand{\PROOF}{\noindent {\bf Proof}: }
\newcommand{\REF}[1]{[\ref{#1}]}
\newcommand{\Ref}[1]{(\ref{#1})}
\newcommand{\dt}{\mbox{\rm   dt}}
\newcommand{\C}{\mathbb{C}}
\newcommand{\full}{\{1, \ldots, n\}}
\newcommand{\phat}{\hat{p}}
\begin{document}
\section{Introduction}\label{S:intro}

  Our main object of study is an algebraic complex $B(n, k)$ that is a filtration of the order complex of the 
Boolean algebra. More precisely, fix positive integers $k$ and $n$. Then define 
  $\Delta_l(n, k) = \{\varnothing \subset C_0 \subset \cdots \subset C_l \subset \{1, \ldots, n\}
  \colon 0 < |C_{i+1} - C_i| \le k 
  \mbox{ for all } -1 \le i \le l \}$. Here $C_{-1} = \varnothing$ and $C_{l+1} = \full$. Let $B_l(n, k)$ be the free abelian group generated by $\Delta_l(n, k)$ over $\C$.
We can now define our main object of study: 
\begin{definition}\label{:B(n,k)} Fix positive integers $n$ and $k$. Then define
  \begin{equation}
    B(n, k) = \bigoplus_{l=0}^{n-2} B_l(n, k).
  \end{equation}
\end{definition}
\begin{example} When $k > n-2$ we get the order complex of the Boolean algebra. When $k=1$ there are only maximal chains.
\end{example}
Let $C = \varnothing \subset C_0 \subset  \cdots \subset C_l \subset \{1, \ldots, n\}$ be a chain in $\Delta_l(n, k)$.
Then define
\begin{equation}\label{E:delta}
  \partial_l(C) = 
  \begin{cases}
     \sum_{i=0}^l (-1)^i \varnothing \subset \cdots \subset {\hat C_i} \subset \cdots  \subset \full, 
	 &\text{if $|C_{i+1} - C_{i-1}| \le k$;}\\
     0, &\text{otherwise.}
  \end{cases}
\end{equation} where ${\hat C_i}$ denotes that this set is omitted from the chain. 
Extend $\partial_l$ linearly to a function on $B_l(n, k)$. Then it is easy to verify that $(B(n, k), \partial)$
is an algebraic complex.  While we have not been able to compute the 
exact dimensions of the homology, we do know exactly when the homology is nonzero. To state our result we need to introduce to following
concept:
\begin{definition}\label{D:type}
If $\varnothing \subset C_0 \subset \ldots \subset C_l \subset \{1, \ldots, n\}$ is
           an element of $\Delta_l(n, k)$ then the {\bf type} of this element is the word 
           $|C_0||C_1 - C_0|\cdots|\full - C_l|$.
\end{definition}
\begin{example} The type of the chain $\varnothing \subset \{1, 3\} \subset \{1, 3, 4\} \subset \{1, 2,3, 4\}$ is the word
$211$. Thus this chain can belong to $B(4, k)$ if $k >1$.
\end{example}

The important observation is the following: 
\begin{lemma} \label{L:grading} If $x$ and $y$ are two elements of $B(n, k)$ of homogenous, yet distinct, type such that
$x+y$ is a cycle, then $x$ and $y$ are individually cycles.
\end{lemma}
Hence type gives a grading to the homology. We can now state our main result on the homology:
\begin{theorem}\label{T:Main}
 Fix $n$ and $k$, then $H_w(B(n,k))$ is nonzero if and only if $w$ is a word containing only 1s and (if $k <n-1$) ks such that
 $w$ begins with 1 and there are no two consecutive ks.
\end{theorem} 
Note then that the homology has the interesting feature of being periodic. 

In the case of the Boolean algebra recall that
the natural action of $S_n$ on the poset $B_n$ defines an action of $S_n$ on the homology. Since this action doesn't 
effect step size of chains we also have an action of $S_n$ on $B(n, k)$, hence on $H_*(B(n, k))$ (we will always consider
complex coefficients when we take homology, so we omit the coefficients from the notation). We can now describe the
main motivation for considering this complex. If we just consider the trivial representation inside of $B(n, k)$ then we
note that since all numbers are equivalent we only need to concern ourselves with the sizes of the various sets. 
Introducing a formal variable $t$ as a place marker we see that the complex is the free abelian group over $\C$ generated
by $t^{i_1}\otimes \cdots \otimes t^{i_l}$, where $0 <i_j \le k$ and the exponents sum to $n$.
 Here the exponents keep track of how the chains grow, 
i.e. $i_1 \cdots i_l$ is the type. Examining the action of the boundary map we find that
\begin{equation} \label{E:Hochreduction}
  H_d(B(n, k))^{S_n} = HH_{d, n}(t\C[t]/(t^{k+1}))
\end{equation}
where $V^G$  denotes the trivial isotypic component of a representation $V$ of $G$ and $HH$ denotes the Hochschild 
homology (see Weibel \cite{Weibel} for information on Hochschild homology). Here the second 
subscript of $HH$ refers to the grading by degree. In fact the right-hand side of ~\ref{E:Hochreduction} is known, see 
Hanlon \cite{HanlonMac}:
\begin{equation} \label{E:HH}
  HH_{d, n}(t\C[t]/(t^{k+1})) = 
  \begin{cases} 
   1, &\text{if $d$ is odd and $\frac{(d+1)(k+1)}{2} = n;$}\\
   1, &\text{if $d$ is even and $\frac{d(k+1)}{2} = n;$}\\
   0, &\text{otherwise.}
  \end{cases}
\end{equation}
In our case this corresponds to $H_{1k1k\cdots1k(1)}(B(n, k))$ being nonzero.

Gerstenhaber and Schack \cite{GS} defined a notion of Hodge decomposition for Hoch-schild homology and Hanlon \cite{Hanlon}
was able to extend this notion to posets that admit a Hodge structure. A Hodge structure on a poset is an action of
$S_l$ on chains of length $l$ which satisfies certain 
relations. Hanlon shows that the following actions define a Hodge structure on $B_n$:
\begin{definition}
  Let $C =\varnothing \subset C_0 \subset \cdots \subset C_l \subset \full$ be a chain of subsets. Fix $1 \le i \le l$. 
  Let $A = C_{i+1} - C_i$. Then for the transposition $\tau = 
  (i, i+1) \in S_l$ define $$\tau C = \varnothing \subset \cdots C_{i-1} \subset C_{i-1} \cup A \subset C_{i+1} \cdots 
  \subset \full.$$
\end{definition}

Note that this action will also work on $B(n, k)$. Moreover the type of $\tau C$ is $\tau w$ where $\tau$ acts on words 
by permuting location. Now that we have described what the Hodge structure of $B(n, k)$ is (even though this is not poset
homology) we need to talk about the Hodge decomposition. Gerstenhaber and Schack defined pairwise orthgonal idempotents in
$\C S_n$, $e_n^{(1)}, \ldots, e_n^{(n)}$ called the Eulerian idempotents \cite{GS}. Unfortunately their definition is not
condusive to computation, see Loday \cite{Loday} for a more concrete definition. Since we have a Hodge structure on 
$B(n, k)$ we can define $$B_l^{(j)}(n, k) = e_l^{(j)} \cdot B_l(n, k).$$ The results of Hanlon (which follow Gerstenhaber
and Schack) show that $$\partial_l \colon B_l^{(j)}(n, k) \rightarrow B_{l-1}^{(j)}(n, k).$$ Thus we have that 
$$H_l(B(n, k)) = \bigoplus_j H_l^{(j)}(B(n, k))$$ and this is the {\bf Hodge decomposition of $H_*(B(n, k))$}. Further the
action of $S_n$ on $B(n, k)$ clearly commutes with the action of $S_l$ on $(l-1)$-chains. Thus each $H^{(j)}(B(n, k))$ is
an $S_n$ representation. Hence we have
\begin{equation}
  H_d^{(j)}(B(n, k))^{S_n} = HH_{d, n}^{(j)}(t\C[t]/(t^{k+1}).
\end{equation}

Unfortunately as Theorem ~\ref{T:Main} shows the complex is not a homology sphere. Thus by employing the theory in Hanlon 
\cite{Hanlon} we are only able to results on the Euler characteristic of each Hodge piece. In particular we have:

\begin{theorem} \label{T:Hodge} Let $\chi_j^{n, k}$ denote the euler characteristic of the $j$ Hodge piece of $B(n, k)$. 
Then we have $$\sum_n (-1)^n \sum_j \lambda^j Z(\chi_j^{n, k}) = - \prod_l (1 + a_l[Z(\epsilon_1) + \cdots + 
Z(\epsilon_k)])^{-\frac{1}{l}\sum{d|l} \mu(d) \lambda^{\frac{l}{d}}}.$$
\end{theorem} 

Here if $f$ is a class function of $S_n$ $Z(f)$ is the cycle indicator, that is $$Z(f) = \frac{1}{n!} \sum_{\sigma \in S_n}
 f(\sigma) Z(\sigma)$$ where $j_i(\sigma)$ denotes the number of $i$ cycles in $\sigma$ and $$Z(\sigma) = 
a_1^{j_1(\sigma)} a_2^{j_2(\sigma)} \cdots a_p^{j_p(\sigma)}.$$ Also the bracket operation is defined as $$A[B] = A[ a_i
\leftarrow B[a_j \leftarrow a_{ij}]]$$ where $\leftarrow$ denotes substitution. For example if $A = a_1^2 + a_2$ and 
$B= a_3+a_1$ then $A[B] = (a_3+a_1)^2 + a_6+a_2$. 


From Theorem ~\ref{T:Hodge} we can obtain the multiplicity of various representations in $\chi_j^{n, k}$:

\begin{corollary} \label{C:Hodge} 
  Fix $k$.
  \begin{enumerate} 
  \item The sign representation only appears in dimension $n-2$ and Hodge piece $n$ with coefficient $(-1)^n$.
  \item The trivial representation appears in Hodge piece $j$ with coefficient $(-1)^{j+1}$ if $n=j(k+1)$ or 
     	  $n=(j+1)(k+1) +1$.
  \item Let $\phi_j^{n, k}$ be the coefficient of the defining representation in $\chi_j^{n, k}$. Then 
        $$ \sum_n (-1)^n \sum_j \lambda^j \phi_j^{n, k}x^n = \frac{(1+x\lambda)(x\lambda - x^k\lambda -x^{k-1}\lambda +
        x^{2k-2}\lambda - 1 +2x -x^2)}{(1-x)^2(1+x^{k+1}\lambda)}.$$
  \item Let $\chi_{n, k}$ denote the Euler characteristic on $B(n, k)$. Fix $k$. Then $$\sum_n (-1)^n Z(\chi_{n, k}) = 
        \frac{1}{1 + (Z(a_1 + a_2 + \cdots + a_k))}.$$
  \end{enumerate} 
\end{corollary}


The rest of the paper is organized as follows: section 2 proves Theorem ~\ref{T:Hodge} and ~\ref{C:Hodge}, section 3 
introduces the spectral sequence that is fundamental to proving Theorem ~\ref{T:Main} and section 4 proves this theorem.

\begin{thebibliography}{9}
\bibitem{GS}
  M. Gerstenhaber and S.D. Schack,
  \emph{A Hodge-type decomposition for commutative algebra cohomology,}
  J. Pure Appl. Algebra {\bf 48} (1987), 229-247.
\bibitem{HanlonMac}
  P. Hanlon,
  \emph{Cyclic homology and the Macdonald conjectures,}
  Invent. Math. {\bf 86} (1986), 131-159.
\bibitem{Hanlon}
  P. Hanlon,
  \emph{Hodge Structure on Posets}
\bibitem{Loday}
  J. L. Loday,
  \emph{Partition eul\'eriene et op\'erations en homologie cyclique.}
  C. R. Acad. Sci. Paris S\'er. I Math. {\bf 307} (1988), 283-286.
\bibitem{Weibel}
  C. Weibel,
  \emph{An introduction to homological algebra,}
  Cambridge University Press, UK, 1994.

\end{thebibliography}
\end{document}

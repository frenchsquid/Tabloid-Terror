\documentclass{elsart}
\usepackage{amsmath,amssymb, setspace}
%include these lines if you want to use the LaTeX "theorem" environments
\newtheorem{theorem}{Theorem}[section]
\newtheorem{definition}[theorem]{Definition}
\newtheorem{lemma}[theorem]{Lemma}
\newtheorem{corollary}[theorem]{Corollary}
\newtheorem{example}[theorem]{Example}
\newtheorem{proposition}[theorem]{Proposition}
%include lines like this if you want to define your own commands 
%to save typing
\newcommand{\PROOF}{\noindent {\bf Proof}: }
\newcommand{\REF}[1]{[\ref{#1}]}
\newcommand{\Ref}[1]{(\ref{#1})}
\newcommand{\dt}{\mbox{\rm   dt}}
\newcommand{\C}{\mathbb{C}}
\newcommand{\full}{\{1, \ldots, n\}}
\newcommand{\phat}{\hat{p}}
\newcommand{\GC}{\varnothing \subset C_0 \subset \cdots \subset C_l \subset \{1, \ldots, n\}}


\begin{document}
\begin{frontmatter}
\title{Stability of the homology of a filtered Boolean algebra}

\begin{abstract}
 In this paper we examine the homology of a certain quotient, $B(n, k)$ of the order complex of a Boolean algebra. This
quotient also has an interpretation as a generalizatin of the Bar complex of a truncated polynomial ring. The main result
is that the homology obeys a stability property. That is as $n$ and $k$ grow large the homology vanishes outside of two 
dimensions. 

Because this complex is constructed from the Boolean algebra there is a natural symmetric group action. We show
one can consider the isotypic components of this complex as a chain complex, $C_*(\lambda, k)$ of semi-standard Young
tableaux of a fixed shape $\lambda$, with a few other restrictions. The complexes $C_*(\lambda, k)$ are crucial to our
proof of stability.

\end{abstract}

\author{Scott Kravitz}
\address{skravitz@umich.edu, Department of Mathematics, University of Michigan, Ann Arbor, MI 48109-1003} 
\begin{keyword}
Poset homology, Boolean algebra, rank-selected subposet, Hodge structure.
\end{keyword}

\date{\today}     


\end{frontmatter}

%--------------------------------------------------------------------------------------------------------------------------
\section{Introduction}\label{S:intro}

  Our main object of study is an algebraic complex $B(n, k)$ that is a filtration of the order complex of the 
Boolean algebra. More precisely, fix positive integers $k$ and $n$. Then define 
  $\Delta_l(n, k) = \{\varnothing \subset C_0 \subset \cdots \subset C_l \subset \{1, \ldots, n\}
  \colon 0 < |C_{i+1} - C_i| \le k 
  \mbox{ for all } -1 \le i \le l \}$. Here $C_{-1} = \varnothing$ and $C_{l+1} = \full$. Let $B_l(n, k)$ be the vector 
   space with basis $\Delta_l(n, k)$ over $\C$.
  We can now define our main object of study: 
\begin{definition}\label{:B(n,k)} Fix positive integers $n$ and $k$. Then define
  \begin{equation}
    B(n, k) = \bigoplus_{l=0}^{n-2} B_l(n, k).
  \end{equation}
\end{definition}
We must make this into a complex by defining a boundary operator. Suppose $C = \varnothing \subset C_0 \subset  \cdots
\subset C_l \subset \{1, \ldots, n\}$ is a chain in $\Delta_l(n, k)$.
Then define $\partial_l(C)$ to be 
\begin{equation}\label{E:delta}
  \begin{cases}
     \sum_{i=0}^l (-1)^i \varnothing \cdots \subset {\hat C_i} \subset \cdots \full, 
         &\text{if $|C_{i+1} - C_{i-1}| \le k$;}\\
     0, &\text{otherwise.}
  \end{cases}
\end{equation} where ${\hat C_i}$ denotes that this set is omitted from the chain. 
Extend $\partial_l$ linearly to a function on $B_l(n, k)$. Then it is easy to verify that $(B(n, k), \partial)$
is an chain complex.  

\begin{example} When $k > n-2$ we get the order complex of the Boolean algebra. When $k=1$ there are only maximal chains.
\end{example}


In the case of the Boolean algebra recall that
the natural action of $S_n$ on the poset $B_n$ defines an action of the symetric group $S_n$ on the homology. Since this action doesn't 
affect the step size of chains we also have an action of $S_n$ on $B(n, k)$, and hence on $H_*(B(n, k))$ (we will always 
consider
complex coefficients when we take homology, so we omit the coefficients from the notation). We can now describe the
main motivation for considering this complex. If we consider the trivial representation inside of $B(n, k)$ then we
note that since all numbers are equivalent we only need to concern ourselves with the sizes of the various sets. 
Introducing a formal variable $t$ as a place marker we see that the complex is the vector space over $\C$ generated
by $t^{i_1}\otimes \cdots \otimes t^{i_l}$, where $0 <i_j \le k$ and the exponents sum to $n$.
 Here the exponents keep track of how the chains grow. 



Examining the action of the boundary map we find that:
\begin{equation} \label{E:Hochreduction}
  H_{d-2}(B(n, k))^{S_n} = HH_{d, n}(t\C[t]/(t^{k+1}))
\end{equation}
where $V^G$  denotes the trivial isotypic component of a representation $V$ of $G$ and $HH$ denotes the Hochschild 
homology (see Weibel \cite{Weibel} for information on Hochschild homology). Here the second 
subscript of $HH$ refers to the grading by degree. 

\begin{example}

  It can be shown that $HH_{2,3}(t\C[t]/(t^3)) \cong \C$ and is generated by $t \otimes t^2$. 
  This  
  corresponds to an element of $H_0(B(3, 2))^{S_3}.$ The corresponding chain is:
  $$ \varnothing \subset \{1\} \subset \{1, 2, 3\} + \varnothing \subset \{2\} \subset \{1, 2, 3\} + \varnothing \subset 
     \{3\} \subset \{1, 2, 3\}.$$
\end{example}

In fact the right-hand side of Eq. ~\ref{E:Hochreduction} is known, see 
Hanlon \cite{HanlonMac}:
\begin{equation} \label{E:HH}
  dim(HH_{d, n}(t\C[t]/(t^{k+1}))) = 
  \begin{cases} 
   1, &\text{if $d$ is odd and $\frac{(d+1)(k+1)}{2} = n;$}\\
   1, &\text{if $d$ is even and $\frac{d(k+1)}{2} = n;$}\\
   0, &\text{otherwise.}
  \end{cases}
\end{equation}

Gerstenhaber and Schack \cite{GS} defined a notion of Hodge decomposition for Hochschild homology and Hanlon \cite{Hanlon}
was able to extend this notion to posets that admit a Hodge structure. A Hodge structure defined in \cite{Hanlon} on a 
poset is an action of the symmetric group
$S_l$ on chains of length $l$ which satisfies certain 
relations. Hanlon showed that $B_n$ admits a Hodge structure and it is easy to see that this structure extends to $B(n, k)$.
This action was studied in a previous paper \cite{S1}. 
In this paper we will study the complex itself.

In particular we can compute the homology almost everwhere. Our main result can be expressed as follows:
\begin{theorem} \label{T:main} Fix $n$, $k>1$ and set $x=n-k$. Then the following hold:
  \begin{enumerate}
    \item \label{i:top} The top dimension, $n-2$, of $H_*(B(n, k))$ is zero except for the sign representation.
    \item \label{i:upper} For $x-1 < d < n-2$ we have that $H_d(B(n, k))=0$.
    \item \label{i:x-1} The only representations that appear in $H_{x-1}(B(n, k))$ are those of the form 
          $(k+1, 2^y, 1^{x-1-2y})$ or
          $(k+2, 2^y, 1^{x-2-2y})$. These appear with a multiplicity of $x-2y$ and $x-2y-1$ respectively.
    \item \label{i:stability} Fix $x$. Then for all $n > x/2$ we have that $H_d(B(n, n-x)) = 0$ for $d < x - 1$.
  \end{enumerate}
\end{theorem}

\begin{example}
  To show what we mean by stability, fix $x=4$. Then for $n> 2$ the dimensions, $d$,  such that $H_d(B(n, n-4))$ is nonzero 
  are $3$ and $n-2$. Further
  in dimension $n-2$ only the sign representation appears and in dimension $3$ we have the partitions
  $(n-3, 1, 1, 1)$, $(n-2, 1, 1)$, $(n-3, 2, 1)$ and $(n-2, 2)$ which appear with multiplicity $4$, $3$, $2$, and $1$
  respectively.
  
  Notice that the multiplicities depend only on $x$, and that except for the top row, the partitions are fixed.
\end{example}

Parts~\ref{i:top} and ~\ref{i:upper} were shown in \cite{S1}. Part~\ref{i:x-1} will be shown in Section~\ref{S:x-1} using 
a spectral sequence argument and part~\ref{i:stability} will be shown in Section~\ref{S:stability}. In Section~\ref{S:data}
we present the irreducible components of $H_*(B(8, k))$. 


%--------------------------------------------------------------------------------------------------------------------------
\section{The complex of semi-standard Young tableaux}

In this section we derive a connection between the irreducible components of $H_*(B(n, k))$ and the homology of a chain
complex of tableaux described below. Then we use this complex to better under stand the homology of $B(n, k)$.

To examine the complex $B(n, k)$ it is best to break it into irreducible $S_n$-modules. We begin by recalling some 
definitions.

\begin{definition}
  Fix a standard Young tableaux, $t$. Let $\{t\}$ be the associated tabloid. Then define:
  $$ r_t = \sum_{\sigma \in S_n \colon \sigma(\{t\}) = \{t\}} \sigma,$$
  $$ {\bar c}_t = \sum_{\sigma \in S_n \colon \sigma((\{t\})') = (\{t\})'} \sigma,$$
  $$ e_t = r_T {\bar c_t}.$$
\end{definition}


Any chain, $\GC$,  in $B(n, k)$ can be thought of as a tabloid, by letting row $i$ of 
the tabloid be $C_{i-1} - C_{i-2}$. 

\begin{example} \label{E:chaintooid}
  For example the chain
  $$ \varnothing \subset \{1\} \subset \{1,2,3\} \subset \{1,2,3,4,5\}$$ corresponds to the tabloid
  $$\begin{array}{cc}
    1 &   \\
    2 & 3 \\
    4 & 5 \\
  \end{array}.$$
\end{example}

Recall then that the $\C S_n$-module, $M^\mu$,  of tabloids of shape $\mu$ is isomorphic
to the $\C S_n$-module, $T_{\lambda, \mu}$, of tableaux of shape $\lambda$ and type $\mu$, for any $\lambda$ (see Sagan 
\cite{Sagan}). Thus we have that as a $\C S_n$-module:
$$ B(n, k) \cong \bigoplus_{\mu \models n} M^\mu \cong \bigoplus_{\mu \models n} T_{\lambda, \mu}$$ where the sum is over 
all compositions of $n$ with no part larger than $k$ and some fixed $\lambda$. 

Recall the following fact:

\begin{theorem}[see \cite{Sagan} section 2.10]
  Fix a SYT, $t_0$,  of shape $\lambda$. Then the $\lambda$-isotypic component of $T_{\lambda, \mu}$ is generated, as an
  $S_n$-module by elements of the form $\bar{c}_{t_0} r_T T$ where $T$ is a SSYT. 
\end{theorem}

We will now examine the action of the boundary operator on one of these components. Recall that to construct the
map from $M^\mu$ to $T_{\lambda, \mu}$ we first pick a tableau, $t$,  of shape $\lambda$ and type $1^n$. Then the 
symmetric group acts on a tableau, $s$,  in $T_{\lambda, \mu}$ by permuting the boxes of $s$ according to the labeling of
$t$. The boundary operator acts by indentifying consecutive numbers and as such if $t$ is a semistandard Young tableaux, 
then the terms of $\partial(t)$ will still be weakly increasing in rows. 

\begin{example}
  The tabloid in Example~\ref{E:chaintooid} in an element of $M^{122}$. Fix $\lambda=311$ and 
  $$t = \begin{array}{ccc} 1 & 2 & 3 \\ 4 & 5 & \\ \end{array} .$$ Then the corresponding element of $T_{122, 311}$ is
  $$s = \begin{array}{ccc} 1 & 2 & 2 \\ 3 & 3\\ \end{array} ,$$ and 
  $$\partial(s) = \begin{array}{ccc} 1 & 1 & 1\\ 2 & 2\\ \end{array}  - \begin{array}{ccc} 1 & 2 & 2 \\ 2 & 2\\ 
    \end{array} .$$
  So notice that tableaux on the right-hand side still increase weakly in the rows.
\end{example}

However a term of $\partial(t)$ may violate the
increasing column condition weakly, but when ${\bar c}_t$ is applied to such a term the 
identical elements of the column will switch, thus resulting in zero. Hence if $T \in SSYT(\lambda)$ we get that

$$\partial({\bar c}_{t_0} T) = {\bar c}_{t_0} S$$

for some other $S \in SSYT(\lambda)$.


Thus we can define a new complex, 
$C_*(\lambda, k)$,  which consists of 
all semistandard Young tableaux of shape $\lambda$ such that no number appears more than $k$ times. Further we have
$$H_*(C_*(\lambda, k)) = <S^\lambda, H_*(B(n, k))>$$ where $S^\lambda$ is the Specht module of shape $\lambda$. 


\begin{example}
  Here we will compute the homology of $C_*(31,2)$.
  The tableaux possible are
         $$\begin{array}{ccc} 1 & 1 & 2 \\ 2&&\\ \end{array}, \begin{array}{ccc} 1 & 1 & 2 \\ 3&&\\ \end{array},
           \begin{array}{ccc} 1 & 1 & 3 \\ 2&&\\ \end{array}, \begin{array}{ccc} 1 & 2 & 2 \\ 3&&\\ \end{array},
           \begin{array}{ccc} 1 & 2 & 3 \\ 2&&\\ \end{array}, \begin{array}{ccc} 1 & 2 & 3 \\ 3&&\\ \end{array},$$
         $$\begin{array}{ccc} 1 & 3 & 3 \\ 2&&\\ \end{array}, \begin{array}{ccc} 1 & 2 & 3 \\ 4&&\\ \end{array},
           \begin{array}{ccc} 1 & 2 & 4 \\ 3&&\\ \end{array}, \begin{array}{ccc} 1 & 3 & 4 \\ 2&&\\ \end{array}.$$
         Thus we get that $\partial_1 = \left( \begin{array}{cccccc}-1 & -1 & 0 & 0 & 1 & 0 \\ 
         \end{array} \right) $ and $\partial_2 = \left(\begin{array}{ccc} 1 & 0 & 0\\ 0 & 1 & 0 \\ -1 & 0 & 0 \\ 0 & -1 & -1\\
         1 & 1 & 0\\ 0 & 0 & 1\\ \end{array} \right) $. Thus we get that $H_0(C_*(31,2)) \cong H_2(C_*(31,2)) \cong 0$ and 
         dim($H_1(C_*(31,2))) = 2$.

\end{example}

%--------------------------------------------------------------------------------------------------------------------------
\section{Dimension $x-1$} \label{S:x-1}

As mentioned above, in the top dimension, $n-2$, of $B(n, k)$ there is exactly the sign representation. However there is
another dimension that acts like the top dimension, $x-1$ (where $x=n-k$) in the sense that $H_d(B(n, k)) = 0 $ for 
$ x-1 < d < n-2$ as mentioned above, but $H_{x-1}(B(n, k)) \ne 0$. In this section we will compute the homology in this 
dimension by using a spectral sequence argument to compute $H_{x-1}(C_*(\lambda, k))$.

First we state the following proposition from an earlier paper:

\begin{proposition}[ See \cite{S1}] \label{P:uppertri} Fix $n$ and $k$, and set $x=n-k$. 
  Except for the sign representation, which occurs with multiplicity 1 in
  dimension $n-2$, the homology of $B(n, k)$ vanishes above dimension $x-1$. 
\end{proposition}

Now we begin our proof by first showing that for most choices of $\lambda$, $H_{x-1}$ $(C_*(\lambda, k)) = 0$. 

\begin{theorem}\label{T:topdim} Fix $n$ and $k > 1$ and set $x=n-k$. Unless $\lambda$ is of the form $(k+1, 2^y, 1^{x-1-2y})$ or 
$(k+2, 2^y, 1^{x-2-2y})$ then $H_{x-1}(C_*(\lambda, k))=0$. 
\end{theorem}

\begin{pf}
  We begin by using a spectral sequence. If $t$ is a semi-standard Young tableau then let $m(t)$ denote the maximal entry
  in $t$ and let $F(t)$ denote the multiplicity of $m(t)$
  in $t$. Notice then that since the boundary operator always combines numbers, that after applying
  the boundary operator we always have the same or more number of maximal elements (though the actual maximal element 
  may change).
  Hence we get:  $$F(\partial(t)) \ge F(t).$$  Thus $F$ gives a filtration on $C_*(\lambda, k)$ for any $\lambda$, any $k$ so
  we can consider the associated spectral sequence. 

  The next step is to compute the $E^1$ term of the spectral sequence, that is to compute the homology of the complex
  $(C_*(\lambda, k), \partial_0)$ where $\partial_0$ is the part of the boundary operator that does not change the 
  filtration index. Since the filtration index only measures the number of maximal elements of $t$, $\partial_0$ can adjoin
  any two numbers except for $m(t)$ and $m(t)-1$. Thus the blocks of $t$ containing maximal elements are never
  changed in any way by $\partial_0$. Further if we removed these blocks we would still get a semi-standard Young tableau
  since we can always choose a maximal element that lies in an outside corner and then do the same for the tableau we 
  have left. What we have removed when we do this is a horizontal strip. Then $\partial_0$ acts the same as $\partial$ on 
  what we have left. Hence we get:
  \begin{equation}\label{E:E1term}
    E^1(C_*(\lambda, k)) \cong \bigoplus H_{x-2}(C_*(\mu, k)).
  \end{equation}
  Where we sum over all $\mu \subset \lambda$ such that $\lambda - \mu$ is a horizontal strip with size less than or equal to $k$.

  If we remove a horizontal 
  strip and get $ \mu =1^n$, we started with a hook shape, so this is a possible case where $H_{x-2}(C_*(\lambda,k)) \ne 0$.

  Otherwise we apply Proposition~\ref{P:uppertri} to the right hand side of Eq.~\ref{E:E1term}. Letting $s=|\lambda - \mu|$
  we wish to know if 
  $$ x-2=n-k-2 >|\mu| -k -1=n-s-k-1.$$ We get that the solution is $$s > 1.$$ 

  Hence if we remove more than one block, the
  homology vanishes so we need only sum over $|\lambda - \mu| = 1$ in Eq.~\ref{E:E1term} (except for the hook  
  shape case). 

  Now we set up the induction argument. We will induct on $n$, so assume the theorem is true for all $n < N$ 
  (letting $X=N-k$). Then for most
  shapes removing an outside corner in Eq.~\ref{E:E1term} will result in a shape of size $N-1$ which is not on the
  list in the theorem. Thus the right-hand side is zero and we are done. Since we can only remove $k$ boxes, the only
  hook to get to the sign representation is a shape where the top row is of size $k+1$ or less. 
  The only shapes that we have to worry about then
  are: 
  \begin{enumerate}
    \item Any hook shape of size $N$ where the first row is of size less than $k+1$.
    \item $(k+3, 2^y, 1^{X-3-2y})$
    \item $(k+1, 3, 2^y, 1^{X-4-2y})$
    \item $(k+2, 3, 2^y, 1^{X-5-2y})$
  \end{enumerate}

  So first we eliminate the hook case: $(a, 1^{N-a})$. 
  We start with a shape where the first row is less than $k+1$ or
  the first row is more than $k+2$. In the first case the partitions on the right hand side of Eq.~\ref{E:E1term} 
   become: $1^{N-a}$ and $1^{N-a-1}$. We got to these shapes
  by removing $a$ [resp. $a+1$] $m$'s. There is only one generator of the sign homology and this is in the top dimension.
  Thus $H_{X-2}(1^y)$ is nonzero only when $X=y$. That will occur only if $X-2 = N-a$ or $X-2 = N-a-1$, that is only if 
  $a= k+1$ or $a=k+2$. To summarize for latter use we have:
  \begin{equation} \label{E:smallhook}
    H_y(a, 1^{n-a}) \cong 0 \mbox{ if $a < k+1$ and $y \ne n-2$}
  \end{equation}

  Now we can move on to the next cases. In all of these cases when we apply Eq.~\ref{E:E1term} there is one 
  important commonality: there is a single term on the right hand side of the equation. This means that in succesive terms
  of the spectral sequence elements of the kernel of a $\partial_i$ can result from cancellation. That is because every 
  tableau has a single maximal element, $m(t)$, in the same place. Thus since $\partial_i$ will only combine that maximal element
  with elements that are one less, the effect of taking $\partial_i$ is the same as placing an $m(t)-1$ element there 
  instead. So if no cancellation occured before, it won't occur now.

  So any cycles that survive must be caused either by (1) violated column strictness or (2) exceeding $k$. To argue against
  these we need to keep careful track of where the maximal elements are, so the following lemma is useful:

  \begin{lemma}\label{L:toprow1}
    Fix $n$ and $k > 1$, set $x=n-k$. Suppose $t$ is a homology generator with maximal element $m=m(t)$ in either 
    $H_{x-1}(C_*((k+1, 1^{n-k-1}),k))$ or $H_{x-1}(C_*((k+2, 1^{n-k-2}),k))$. Then the top row of $t$ begins with a $1$ 
    and then $k$ of some number followed by another number in the $k+2$ case. 
  \end{lemma}
  \begin{pf}
    We proceed by induction on $n$. So assume the lemma is true for $n, k < N$. The base case is easily verified by hand.
    Now consider Equation~\ref{E:E1term}. We first take the partition $(k+1, 1^{n-k-1})$. The equation becomes: 
    \begin{equation}\label{E:E1term-hookcase-1}
      H_{x-2}(C_*((1^{n-k}),k)) \oplus 
      H_{x-2}(C_*((k, 1^{n-k-1}),k)) \oplus H_{x-2}(C_*((k+1, 1^{n-k-2}),k)).
    \end{equation}

    Since we obtained this equation by removing all the $m$'s from the tableau, we see that in the case of the  sign 
    representation we are done, because there $k$ $m$'s appear in the first row (note that we don't have $1^{n-k-1}$ since in the  sign case we've 
    taken away $k+1$ $m$'s, which can't happen, so that term is zero). In the third term of the equation the induction 
    hypothesis applies. That just leaves the $(k, 1^{n-k-1})$ term. However by 
    Equation~\ref{E:smallhook} the homology of this is zero since $x-2 = n-k -2 < n-3 = $ the size of $(k, 1^{n-k-1})$
    ($k>1$ by hypothesis).

    Next we examine the $(k+2, 1^{n-k-2})$ case. Here the equation becomes: 
    \begin{equation}\label{E:E1term-hookcase-2} 
      H_{x-2}(C_*((k+1, 1^{n-k-2}),k)) \oplus H_{x-2}(C_*((k+2, 1^{n-k-3}),k)).
    \end{equation}
     
    In both these cases the induction hypothesis applies, so we are done.
\qed  
\end{pf} 
  

  First we will deal with the possibility that we violate column strictness. This can only happen with case 3 and 4. Here
  there is, after applying the appropriate $\partial_i$,  a $m(t)-1$ in the third box of the second row and in the third box 
  of the first row there also a $m(t)-1$. Thus by Lemma~\ref{L:toprow1} there is a $m(t)-1$ in the second box of the first row, 
  hence since the second box of the first row was not a $m(t)$ and thus not changed by $\partial_i$, there is nothing that can

  go there without violating column strictness of the second column. So this case cannot happen.

  That just leaves us with the case of exceeding $k$. In cases 2 and 4 this cannot happen because by Lemma~\ref{L:toprow1} 
  the only repeated number is not $m(t)$ in the $k+2$ case. In case 3 we can have $m(t)$ be repeated,
  but then this will reduce to the case of violating column strictness, which was already covered. So we have finished the induction
  argument.
\qed
\end{pf}

\begin{corollary} \label{C:firstcolumn}
  If $ \lambda \ne 1^n $ is such that $\lambda_1 < k$, then $H_*(C_*(\lambda, k))=0$.
\end{corollary}
\begin{pf}
  This is a simple application of Eq.~\ref{E:E1term}
\qed
\end{pf}

Now that we have  shown that only certain shapes can appear in dimension $x-1$, we will show that they indeed do appear.
In particular we exactly describe the generators of homology:


\begin{definition} Fix $n$ and $k > 1$ and set $x=n-k$. Fix a partition $\lambda$ of the form $(k+1, 2^y, 1^{x-1-2y})$ or 
$(k+2, 2^y, 1^{x-2-2y})$. Then for $1 \le a \le x+1$ let $t$ be any tableau of shape $\lambda$ where the first row consists of a $1$ 
followed by $k$ a's. Then define $t_a = \sum sgn ( \sigma ) \sigma (t)$ where the sum is over all permutations in $S_{x+1}$ which 
sends $t$
to a valid SSYT and fixes $1$ and $a$. Note $t_a$ is only well defined up to $\pm 1$.
\end{definition}

\begin{theorem} Fix $n$ and $k > 1$ and set $x=n-k$. Fix a partition $\lambda$ of the form $(k+1, 2^y, 1^{x-1-2y})$ then for $y+2 \le a \le
  x-y+1$ the set of $t_a$ are a set of homology generators. If $\lambda$ is of the form $(k+2, 2^y, 1^{x-2-2y})$ then for $y+2 \le a \le 
  x-y$ the set of $t_a$ are a set of homology generators in dimension $x-1$. Hence

  \begin{equation}\label{E:dim1}
    dim(H_{x-1}(C_*((k+1, 2^y, 1^{x-1-2y}),k) )) = x-2y
  \end{equation}
  \begin{equation}\label{E:dim2}
    dim(H_{x-1}( C_*((k+2, 2^y, 1^{x-2-2y}),k) )) = x-2y-1.
  \end{equation}

\end{theorem}

\begin{example}
  Let $n=5$ and $k=2$ (so $x=3$). Then we get that the generators of $H_2(C_*(41,2))$ are:
  $$t_2 = \begin{array}{cccc} 1 & 2 & 2 & 3 \\ 4 \end{array}
        - \begin{array}{cccc} 1 & 2 & 2 & 4 \\ 3 \end{array}$$ and
  $$t_3 = \begin{array}{cccc} 1 & 3 & 3 & 4 \\ 2 \end{array}.$$
\end{example}
\begin{pf}
  First we need to show that the $t_a$ exist. Note that if $a$ is too large relative to $y$ we won't be able to create that intial tableaux. 
  Since there are $y$ boxes in the second column below the first row, we need there to be $y$ numbers larger than $a$, hence the $x+1-y$ upper bound. In the 
  second case there is an extra number above a at the end of the first row that is needed, hence the upper bound in that case is $x+1-y-1$.
  

  Next we show that $t_a$ are cycles. So applying the boundary operator to  $t_a$ we will get a sum of tableaux where, besides 
  a's in the top row, there is only one number repeated in each tableau. That one number came from a pair of consecutive numbers in $t_a$.
  If these do not appear in the same row, than there is another term in $t_a$ where they appear swapped, hence with opposite sign. 

  Now the claim is that the lower bound $y+2 \le a$ guarantees that no two consecutive numbers appear in the same row (after perhaps the 
  first row). Note that everything not in the first column must be larger than $a$. Any number less than $a$
  must go in the first column. Since there are at least $y+1$ numbers less than $a$, the first $y+1$ boxes in the first column are less 
  than $a$, while there are only $y+1$ boxes in the second column total, all of which are larger than $a$ (except for the first box in the
  column). Thus no row can have consecutive pairs of numbers.

  Next we will demonstrate that these $t_a$ are the homology generators. We proceed by induction on $n$. The base case is easy to verify by 
  hand. First  consider a partition of the form $\lambda = (k+1, 2^y, 1^{x-1-2y})$ and the same spectral sequence used in 
  Theorem~\ref{T:topdim}. We have 
  \begin{equation} \label{E:generator1} 
    E_{x-1}(C_*(\lambda,k)) \cong H_{x-2}(C_*(\mu,k)) \oplus H_{x-2}(C_*(\nu,k))
  \end{equation}
  where $\mu = (k+1, 2^{y-1}, 1^{x-2y})$ and
  $\nu = (k+1, 2^y, 1^{x-2-2y})$. By the induction hypothesis we have that $t^{\mu}_{y+1}, \ldots, t^\mu_{x-y+1}$ are a set of homology 
  generators for $H_{x-2}(C_*(\mu,k))$ and that $t^{\nu}_{y+2}, \ldots, t^\nu_{x-y}$ are a set of generators for $H_{x-2}(C_*(\nu,k))$. The isomorphism in
  Eq.~\ref{E:generator1} consists of attaching $x+1$ at the end of column 2 of $\mu$ and at the end of column 1 in $\nu$. 

  Recall then that $\partial_1$ combines $x+1$ and $x$ if there is only one $x$ in the tableau. In the case of $y \ne 0$ there is only one 
  $x$ in the tableau by construction. So applying $\partial_1$ to a term $s$ of a $t^\mu_a$ [resp. $t^\nu_a$] will result in zero only if 
  column 2 [resp. 1] of $s$ ends in an $x$, otherwise it results in both columns ending in an $x$. Thus for $\partial_1(t^\alpha_a)=0$ we
  need that every term has that condition. In the case $\alpha=\mu$ this forces $x$ to be in the second column for every term, thus there is
  no number larger than $a$ to swap $x$ with in the first column. Hence there are exactly $y$ numbers larger than $a$, so $a=x-y+1$.
  In the case $\alpha=\nu$ there is always a number in the second column to swap $x$ with, so $\partial_1(t^\nu_a) \ne 0$.
  In fact for every term $s$ of $t^\nu_a$ not killed by $\partial_1$ there is a corresponding term $s'$ of $t^\mu_a$ not killed by 
  $\partial_1$ that
  has the same image. Simply take the image of $s$ and remove the $x$ at the end of column 2 to get $s'$. Hence 
  $\partial_1(t^\nu_a - t^\mu_a) = 0$.  Further note that $t^\nu_a-t^\mu_a = \pm t^\lambda_a$ and $t^\mu_{x-y+1} = t^\lambda_{x-y+1}$.
  So we have computed the kernel of $\partial_1$, and there is no image. Further if $ y> 1$ the sequence terminates here, otherwise 
  $\partial_k$ adjoins the $k$ $x$ in $t^\mu_x$ to the $x+1$ to get zero. In either case, the limit of the sequence has $t^\lambda_{y+2}, 
  \ldots, t^\lambda_{x-y+1}$ as the homology generators.
  Thus we have taken care of the case where $y \ne 0$.

  If $y=0$ then we have 
  \begin{equation} \label{E:gen2} 
    E_{x-1}(C_*((\lambda,k)) \cong H_{x-2}(C_((\mu,k)) \oplus H_{x-2}(C_*(\gamma,k))
  \end{equation}
  where $\mu = (k+1, 1^{x-2})$ and $\nu = 1^x$. Then $t^\mu_2, \ldots, t^\mu_x$ are homology generators for $H_{x-2}(\mu)$. Since 
  $\mu$ is a hook, applying
  $\partial_1$ to $t^\mu_a$ for $a<x$ gives zero, since $x$ must appear above $x+1$ in the first column. Applying $\partial_1$ to 
  $t^\mu_x$ gives zero since there is more than one $x$ to add to $x+1$. Let $T$ be such that  $H_{x-2}(C_*(\nu,k)) = <T>$. Then under the 
  isomorphism of Equation~\ref{E:gen2} $T$ is mapped to $T$ with $k$ $x+1$s added in the top row. Hence $\partial_1(T) = 0$. Note that 
  again there is no image. Thus the $E^2$-term is generated by the images of $t^\mu_2, \ldots, t^\mu_x, T$. Finally $\partial_k(t^\mu_x)
  = 0$, so $H_{x-1}(C_*(\lambda,k))$ is generated by the images of $t^\mu_2, \ldots, t^\mu_x, T$. Note that the image of $t^\mu_a$ is $t^\lambda_a$
  and that the image of $T$ is $t^\lambda_{x+1}$. 

  Now we need to consider a partition of the form $\lambda = (k+2, 2^y, 1^{x-2-2y})$. We have 
  \begin{equation} \label{E:gen3} 
    E_{x-1}(C_*(\lambda,k)) \cong H_{x-2}(C_*(\mu,k)) \oplus H_{x-2}(C_*(\nu,k)) \oplus H_{x-2}(C_*(\gamma,k))
  \end{equation}
 
  where $\mu = (k+2, 2^{y-1}, 1^{x-2y-1})$, $\nu = (k+2, 2^y, 1^{x-3-2y})$, and $\gamma$ $= (k+1, 2^y,$ $1^{x-2-2y})$. By the induction 
  hypothesis we have that $t^{\mu}_{y+1}, \ldots, t^\mu_{x-y}$ are a set of homology generators for $H_{x-2}$ $(C_*(\mu,k))$, that 
  $t^{\nu}_{y+2}, \ldots, t^\nu_{x-y-1}$ are a set of generators for $H_{x-2}$ $(C_*(\nu,k))$, and that $t^\gamma_{y+2}, \ldots, t^{\gamma}_{x-y}$ are
  a set of homology generators for $H_{x-2}$ $(C_*(\gamma,k))$. The isomorphism in
  Equation~\ref{E:gen3} consists of attaching an $x+1$ at the end of column 2 of $\mu$, an $x+1$ at the end of column 1 in $\nu$ and an
  $x+1$ at the end of row 1 of $\gamma$. 

  In the case of $y \ne 0$ there is only one 
  $x$ in the tableau by construction. If $y=0$ not that the sign representation no longer appears in the $E^1$ term.
  So applying $\partial_1$ to a term $s$ of a $t^\mu_a$ [resp. $t^\nu_a$] will result in zero only if 
  column 2 [resp. 1] of $s$ ends in an $x$, otherwise it results in both columns ending in an $x$. Applying $\partial_1$ to $t^\gamma_a$ 
  will never result in zero. As above for every term $s$ (which does not have an $x$ in the top row) of $t^\nu_a$ not killed by 
  $\partial_1$ there is a corresponding term 
  $s'$ of $t^\mu_a$ not killed by $\partial_1$ that has the same image, simply take the image of $s$ and remove the $x$ at the end of 
  column 2 to get $s'$. For those terms $s$ (in either $t^\mu_a$ or $t^\nu_a$) which have an $x$ in the top row there is a corresponding
  term $s'$ of $t^\gamma_a$ not killed by $\partial_1$ that
  has the same image, simply take the image of $s$ and remove the $x$ at the end of row 1 to get $s'$. 

  However we now have three terms to worry about and the cancellation might not be simple. We need to keep track of signs better to show 
  that 
  \begin{equation}\label{E:gen4}
    \partial_1(t^\mu_a -t^\nu_a + t^\gamma_a) = 0.
  \end{equation}
  Recall that the $t_a$ are only defined up to sign to begin with. So assume that our
  conventions are such that the requisite terms of $t^\mu_a$ and $t^\nu_a$ cancel in Eq.~\ref{E:gen4}. We now need to show that with 
  such a convention the necessary terms of $t^\mu_a$ and $t^\nu_a$ cancel with the terms of $t^\gamma_a$. Notice that for a term $s$ in 
  $t^\gamma_a$ to cancel with a term $s'$ in $t^\mu_a$ there must be an $x$ in the 2nd column of $s$, while to cancel with a term $s''$ of 
  $t^\nu_a$ there must be an $x$ in the 1st column of $s$. Thus the terms canceling with the $t^\mu_a$ have opposite sign to those 
  canceling with the $t^\nu_a$, so Eq.~\ref{E:gen4} is valid for $y+2 \le a \le x-y-1$. Hence $$\ker \partial_1 = < 
  t^\mu_{y+2} -t^\nu_{y+2} + t^\gamma_{y+2}, \ldots, t^\mu_{x-y-1} -t^\nu_{x-y-1} + t^\gamma_{x-y-1}, t^\mu_{x-y}+ t^\gamma_{x-y}>.$$
  Further note that $t^\nu_a-t^\mu_a+ t^\gamma_a = \pm t^\lambda_a$ and $t^\mu_{x-y}+t^\gamma_{x-y} = t^\lambda_{x-y}$.
  So we have computed the kernel of $\partial_1$, and there is no image. If $y>0$ the sequence terminates here and we are done. Otherwise
  $\partial_k(t^\mu_x)= 0$ by exceeding $k$, and everything else has already stablilized at this point, so we are done.
\qed
\end{pf}  



%--------------------------------------------------------------------------------------------------------------------------

\section{Stability} \label{S:stability}

In this section we prove that the homology of $B(n, k)$ satisfies a stability result. In the previous section we were able
to obtain precise results concering the homology generators in dimension $x-1$. We will not be able to do that now, but 
instead we will be able to show that the only nonzero elements of homology are "new" cycles, i.e. cycles that are not 
cycles in $B_n$.  Our method will be to analyze the type of a chain, as described follows:

\begin{definition} 
  If $C=\GC \in B(n ,k)$ then define the type of $C$ to be the word $|C_0||C_1 - C_0| \cdots |\{1, \ldots, n\} - C_l|$.
\end{definition}

We will then show that any cycle is homologous to a cycle which satsifies the following condition:

\begin{definition}
  We say that $C=\GC \in B(n, k)$ satisfies the $1|k$ condition if type($C$) contains only $1$'s and $k$'s. Further a chain
  $z \in B_l(n, k)$ is said to satisfy the $1|k$ condition if every term in $z$ does.
\end{definition}

Before we begin the proof, it will be convenient to establish some notation:
\begin{definition}  
  \begin{enumerate}
        \item If $w$ is a word, let $|w|$ denote the sum of its letters.
        \item Let $w$ and $v$ be words such that $|w| = |v|$ and the length of $w$ equals the length of $v$. Then say
              that $w < v$ if $w$ violates the $1-k$-condition either before $v$ or the number that violates the condition
              is less. 
        \item Let $N = \{1, \ldots, n \}$.
     \end{enumerate}
\end{definition}

  \begin{theorem} Fix $n$ and $k$. Any cycle in $B(n, k)$ is homologous to either a cycle whose type consists solely of ones and
    k's or zero.
  \end{theorem}
  \begin{pf} 
    This is trivially true if $k \le 2$, so we assume $k > 2$.

Let $z$ be a cycle in $B(n, k)$. Decompose $z$ into parts of homogenous type, $z_1 + \cdots + z_r$. 
    Without loss of generality suppose that $z_1$ violates the $1|k$ condition minimaly with respect to the other $z_i$.
    Let $w$ be the type of $z_1$ and let $w = xay$ where $x$ and $y$ are words and $a$ is the first letter not 
    $1$ or $k$. Finally let $l$ be the index of the place where there is a jump of size $a$. Let $\alpha$ be the last
    number of $x$.

    The goal is to generate an element $b$ of one dimension higher than $z$ such that 
    \begin{enumerate}
      \item $d_{l} (b) = \pm z_1$
      \item $d_i (b) = 0$ for $i < l$
      \item For any term, $s$,  of $\partial(b)$ we have $type(s) > type(z_1)$.
    \end{enumerate}
    There are three cases to consider   
    \begin{enumerate}
      \item[$\alpha=\varnothing$]
        Let $$z_1 = \sum_{i=1}^u C_i (\varnothing \subset C_{1,i} \subset \cdots \subset C_{r,i} \subset 
        N).$$ Let $b_i =\sup(C_{1,i})$
        and let $$b = \sum_{i=1}^u C_i (\varnothing \subset \{b_i\} \subset C_{1,i} \subset 
        \cdots \subset C_{r,i} \subset N).$$ Then clearly $d_1(b) = -z_1$, and the second condition is true
        vacuously. Further the type of any term, $s$,  of $\partial(b)$ starts with a $1$, hence type($s) >$ type($w$).
   
     \item[$\alpha=k$]
        Let $$z_1 = \sum_{i=1}^u C_i (\varnothing \subset C_{1,i} \subset \cdots \subset C_{r,i} \subset 
        N).$$ Hence  
        $|C_{l+1, i} - C_{l, i}| = a$ and $|C_{l, i} - C_{l-1, i}| = k$.  Then let 
        $b_i = \sup(C_{l+1,i}-C_{l, i})$ and let $$b = \sum_{i=1}^u C_i (\varnothing \subset C_{1,i} \subset \cdots C_{l,i} 
        \subset C_{l,i} \cup \{b_i\} \subset C_{l+1, i} \subset \cdots \subset C_{r,i} \subset 
        N).$$ Obviously $d_l(b) = \pm z$. We next need to show that $d_i(b)=0$
        for $i < l$. We will use the following fact for $i< l$: $d_i(z_1)=0.$ The reason is that any term $s$ of $\partial(z)$ 
        that cancels a term of $d_i(z_1)$ would have a $2$ in its type at $i$, hence $type(s) < 
        type(z_1)$ violating the minimality condition. If two terms $a$ and $b$ of $z_1$ cancel in $d_i(z_1)$ then since they have 
        the same type they must agree everywhere except at $C_{i, a}$ and $C_{i, b}$, hence $b_a=b_b$. So these two terms cancel in $d_i(b)$
        so that $d_i(b)=0$ for $i < l-1$. For $d_{l-1}$, this adjoins a jump of size $k$ with a jump of $1$, hence is zero.
        The last condition is obvious as well.
     \item[$\alpha=1$] Let $$z_1 = \sum_{i=1}^u C_i (\varnothing \subset C_{1,i} \subset \cdots \subset C_{r,i} \subset 
        N).$$  Hence  
        $|C_{l+1, i} - C_{l, i}| = a$ and $|C_{l, i} - C_{l-1, i}| = 1$. 
        Again $d_i(z_1)=0$ by minimality. Let us examine 
        \begin{equation} \label{E:stab1}
          d_{l-1}(z_1) = 0.
        \end{equation}
        Here we are removing a set that was a jump of size one. Let $x$ be a term in $z_1$. Then in $z_1$ there is some term $y$ that 
        cancels $x$ in Eq.~\ref{E:stab1}. We will write out $x$ as its jumps, so we write it as a set partition. So 
        $$x= F/a/b_1b_2 \cdots /G$$ where $F$ and $G$ are set partitions themselves and $a$ is the number added at $C_l$, and 
        $\{b_1,b_2 \cdots \}$ is the set added at $C_{l+1}$. The only term that could cancel $x$ at $d_{l-1}$ is the following:
        $$y= F/b_1/ab_2 \cdots/G$$ for some choice of $b_2$. So there is a pairing of terms of $z_1$ along this cancellation. Hence we 
        have
        \begin{equation} \label{E:stab2}
          z_1 = \sum x - y
        \end{equation}
        where $x$ and $y$ are related as described above.  

        Now define $b$ as
        $$ \sum (F/a/b_2/b_1b_3 \cdots/G) - (F/b_1/b_2/ab_3 \cdots/G)$$  
        $$- (F/b_2/a/b_1b_3 \cdots/G) + (F/b_2/b_1/ab_3 \cdots /G)$$
       
        where $b_2$ is the smallest $b_i$ not $b_1$.

        Then $d_l(b)$ is $$\sum (F/a/b_2b_1b_3 \cdots/G) - (F/b_1/b_2ab_3 \cdots/G)$$  $$- (F/b_2/ab_1b_3 \cdots/G) + 
        (F/b_2/b_1ab_3 \cdots /G).$$ Which simplifies to $$ \sum x-y = z_1.$$

        Also $d_{l-1}(b)$ is $$\sum (F/ab_2/b_1b_3 \cdots/G) - (F/b_1b_2/ab_3 \cdots/G)$$ $$- (F/b_2a/b_1b_3 \cdots/G) + 
        (F/b_2b_1/ab_3 \cdots /G).$$ Which simplifies to zero.


        Now we consider $d_i(b)$ for $i < l-1$. If the jump at $i$ or $i+1$ is of size $k$, we do not need to worry. 
        If not, then decompose $F$ as $$K/p/q/L$$ where $p$ is the number added at step $i$. Then since $d_i(z_1)=0$ there is some $x'$ term
        of $z_1$ which cancels with $x$ in $d_i(z_1)$. Then $x'$ must be of the form $$ K/q/p/L/a/b_1 \cdots/G.$$ Also there is a term $y'$
        which looks like $$K/q/p/L/b_1/ab_2 \cdots/G.$$ When we look at the terms corresponding to $x'$ and $y'$ in $b$, we see that the 
        choice of "$b_2$" for the pair $x'$ and $y'$ is the same by construction of $b_2$ as the smallest non-$b_1$ $b_i$. Then putting this
        together we see that there are terms in $b$ corresponding to $x'$ and $y'$ which cancel the terms corresponding to $x$ and $y$ 
        respectively upon application of $d_i$. So $d_i(b) = 0$ for $i < l$.

        Finally note that the last condition holds.
    \end{enumerate}

    Finally note that from out three properties  $type(z \pm \partial(b)) > type(z)$. Hence by induction we are done we can eliminate terms
    that violate the $1|k$-condition until either we get something that satisfies that condition, or perhaps zero.
\qed
  \end{pf}

  \begin{corollary} Fix $n$ and $k$. Then the following hold:

   \begin{enumerate}
    \item $H_d(B(n, k))$ is nonzero only if $n$ can be written as a $(d+2)$-composition with parts $1$ and $k$.
    \item {\bf Stability} Fix $x$. Then for all $n > x/2$ we have that $H_i(B(n, n-x)) = 0$ for $i < x - 1$. 
   \end{enumerate}
  \end{corollary}



%--------------------------------------------------------------------------------------------------------------------------

\section{Data} \label{S:data}
Table~\ref{T:8_2} through Table~\ref{T:8_6} present the multiplicities of the irreducible representations of $S_8$ inside of $H_*(B(8, k))$ for 
various values of $k$. We have used Corollary~\ref{C:firstcolumn} to pare down the table lengths. Also the case $k=1$ is
simply the regular representation in the dimension 6 and $k>6$ is the Boolean algebra.

%--------------------------------------------------------------------------------------------------------------------------
\section{Acknowledgements}
The author thanks his advisor, Phil Hanlon, not only for suggesting the problem, but for many meetings that were both 
useful and encouraging.

%--------------------------------------------------------------------------------------------------------------------------


\begin{table}
\begin{tabular}{cccccccc}
d= & 0 & 1 & 2 & 3 & 4 & 5 & 6  \\
$11111111$ & 0 & 0 & 0 & 0 & 0 & 0 & 1  \\
$2111111$  & 0 & 0 & 0 & 0 & 0 & 0 & 0  \\
$221111$   & 0 & 0 & 0 & 0 & 0 & 0 & 0  \\
$22211$    & 0 & 0 & 0 & 0 & 0 & 0 & 0  \\
$222$      & 0 & 0 & 0 & 0 & 0 & 0 & 0  \\
$311111$   & 0 & 0 & 0 & 0 & 0 & 6 & 0  \\
$32111$    & 0 & 0 & 0 & 0 & 0 & 4 & 0  \\
$3221$     & 0 & 0 & 0 & 0 & 0 & 2 & 0  \\
$3311$     & 0 & 0 & 0 & 0 & 6 & 0 & 0  \\
$332$      & 0 & 0 & 0 & 0 & 3 & 0 & 0  \\
$41111$    & 0 & 0 & 0 & 0 & 0 & 5 & 0  \\
$4211$     & 0 & 0 & 0 & 0 & 6 & 3 & 0  \\
$422$      & 0 & 0 & 0 & 0 & 3 & 1 & 0  \\
$431$      & 0 & 0 & 0 & 0 & 12 & 0 & 0  \\
$44$       & 0 & 0 & 0 & 0 & 4 & 0 & 0  \\
$5111$     & 0 & 0 & 0 & 0 & 6 & 0 & 0  \\
$521$      & 0 & 0 & 0 & 0 & 12 & 0 & 0  \\
$53$       & 0 & 0 & 0 & 0 & 6 & 0 & 0  \\
$611$      & 0 & 0 & 0 & 0 & 9 & 0 & 0  \\
$62$       & 0 & 0 & 0 & 0 & 6 & 0 & 0  \\
$71$       & 0 & 0 & 0 & 0 & 3 & 0 & 0  \\ $8$        & 0 & 0 & 0 & 0 & 0 & 0 & 0  \\
\end{tabular}
\caption[$H_*(B(8,2))$]{$H_*(B(8, 2))$}
\label{T:8_2}
\end{table}

\begin{table}
\begin{tabular}{cccccccc}
d=         & 0 & 1 & 2 & 3 & 4 & 5 & 6  \\
$11111111$ & 0 & 0 & 0 & 0 & 0 & 0 & 1  \\
$311111$   & 0 & 0 & 0 & 0 & 0 & 0 & 0  \\
$32111$    & 0 & 0 & 0 & 0 & 0 & 0 & 0  \\
$3221$     & 0 & 0 & 0 & 0 & 0 & 0 & 0  \\
$3311$     & 0 & 0 & 0 & 0 & 0 & 0 & 0  \\
$332$      & 0 & 0 & 0 & 0 & 0 & 0 & 0  \\
$41111$    & 0 & 0 & 0 & 0 & 5 & 0 & 0  \\
$4211$     & 0 & 0 & 0 & 0 & 3 & 0 & 0  \\
$422$      & 0 & 0 & 0 & 0 & 1 & 0 & 0  \\
$431$      & 0 & 0 & 0 & 0 & 0 & 0 & 0  \\
$44$       & 0 & 0 & 1 & 0 & 0 & 0 & 0  \\
$5111$     & 0 & 0 & 0 & 0 & 4 & 0 & 0  \\
$521$      & 0 & 0 & 0 & 0 & 2 & 0 & 0  \\
$53$       & 0 & 0 & 1 & 0 & 0 & 0 & 0  \\
$611$      & 0 & 0 & 0 & 0 & 0 & 0 & 0  \\
$62$       & 0 & 0 & 1 & 0 & 0 & 0 & 0  \\
$71$       & 0 & 0 & 1 & 0 & 0 & 0 & 0  \\
$8$        & 0 & 0 & 1 & 0 & 0 & 0 & 0  \\
\end{tabular}
\caption{$H_*(B(8,3))$}
\label{T:8_3}
\end{table}

\begin{table}
\begin{tabular}{cccccccc}
d= & 0 & 1 & 2 & 3 & 4 & 5 & 6  \\
$11111111$ & 0 & 0 & 0 & 0 & 0 & 0 & 1  \\
$41111$    & 0 & 0 & 0 & 0 & 0 & 0 & 0  \\
$4211$     & 0 & 0 & 0 & 0 & 0 & 0 & 0  \\
$422$      & 0 & 0 & 0 & 0 & 0 & 0 & 0  \\
$431$      & 0 & 0 & 0 & 0 & 0 & 0 & 0  \\
$44$       & 0 & 0 & 0 & 0 & 0 & 0 & 0  \\
$5111$     & 0 & 0 & 0 & 4 & 0 & 0 & 0  \\
$521$      & 0 & 0 & 0 & 2 & 0 & 0 & 0  \\
$53$       & 0 & 0 & 0 & 0 & 0 & 0 & 0  \\
$611$      & 0 & 0 & 0 & 3 & 0 & 0 & 0  \\
$62$       & 0 & 0 & 0 & 1 & 0 & 0 & 0  \\
$71$       & 0 & 0 & 0 & 0 & 0 & 0 & 0  \\
$8$        & 0 & 0 & 0 & 0 & 0 & 0 & 0  \\
\end{tabular}
\caption[$H_*(B(8,4))$]{$H_*(B(8, 4))$}
\label{T:8_4}
\end{table}

\begin{table}
\begin{tabular}{cccccccc}
d= & 0 & 1 & 2 & 3 & 4 & 5 & 6  \\
$11111111$ & 0 & 0 & 0 & 0 & 0 & 0 & 1  \\
$5111$     & 0 & 0 & 0 & 0 & 0 & 0 & 0  \\
$521$      & 0 & 0 & 0 & 0 & 0 & 0 & 0  \\
$53$       & 0 & 0 & 0 & 0 & 0 & 0 & 0  \\
$611$      & 0 & 0 & 3 & 0 & 0 & 0 & 0  \\
$62$       & 0 & 0 & 1 & 0 & 0 & 0 & 0  \\
$71$       & 0 & 0 & 2 & 0 & 0 & 0 & 0  \\
$8$        & 0 & 0 & 0 & 0 & 0 & 0 & 0  \\
\end{tabular}
\caption[$H_*(B(8,5))$]{$H_*(B(8, 5))$}
\label{T:8_5}
\end{table}

\begin{table}
\begin{tabular}{cccccccc}
d= & 0 & 1 & 2 & 3 & 4 & 5 & 6  \\
$11111111$ & 0 & 0 & 0 & 0 & 0 & 0 & 1  \\
$611$      & 0 & 0 & 0 & 0 & 0 & 0 & 0  \\
$62$       & 0 & 0 & 0 & 0 & 0 & 0 & 0  \\
$71$       & 0 & 2 & 0 & 0 & 0 & 0 & 0  \\
$8$        & 0 & 1 & 0 & 0 & 0 & 0 & 0  \\
\end{tabular}
\caption[$H_*(B(8,6))$]{$H_*(B(8, 6))$}
\label{T:8_6}
\end{table}

\begin{thebibliography}{9}
\bibitem{GS}
  M. Gerstenhaber and S.D. Schack,
  \emph{A Hodge-type decomposition for commutative algebra cohomology,}
  J. Pure Appl. Algebra {\bf 48} (1987), 229-247.
\bibitem{HanlonHodge}
  P. Hanlon, 
  \emph{The Action of $S_n$ on the Components of the Hodge Decomposition of Hoschschild Homology}
  Michigan Math. J. {\bf 37} (1990), 105-124.
\bibitem{HanlonMac}
  P. Hanlon,
  \emph{Cyclic homology and the Macdonald conjectures,}
  Invent. Math. {\bf 86} (1986), 131-159.
\bibitem{Hanlon}
  P. Hanlon,
  \emph{Hodge Structure on Posets,}
  Proc. AMS, to appear.
\bibitem{S1}
  S. Kravitz,
  \emph{The Hodge structure on a filtered Boolean algebra,}
  Submitted.
\bibitem{Loday}
  J. L. Loday,
  \emph{Partition eul\'eriene et op\'erations en homologie cyclique.}
  C. R. Acad. Sci. Paris S\'er. I Math. {\bf 307} (1988), 283-286.
\bibitem{Sagan}
  B. E. Sagan,
  \emph{The Symmetric Group: Representations, Combinatorial Algorithms, and Symmetric Functions}
  Springer-Verlag, NY, 2001
\bibitem{Weibel}
  C. Weibel,
  \emph{An introduction to homological algebra,}
  Cambridge University Press, UK, 1994.

\end{thebibliography}

\end{document}

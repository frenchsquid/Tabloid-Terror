\documentclass{amsart}
\usepackage{amsmath,amsthm,amssymb}
%include these lines if you want to use the LaTeX "theorem" environments
\newtheorem{theorem}{Theorem}[section]
\newtheorem{definition}[theorem]{Definition}
\newtheorem{lemma}[theorem]{Lemma}
\newtheorem{corollary}[theorem]{Corollary}
\newtheorem{example}[theorem]{Example}
%include lines like this if you want to define your own commands 
%to save typing
\newcommand{\PROOF}{\noindent {\bf Proof}: }
\newcommand{\REF}[1]{[\ref{#1}]}
\newcommand{\Ref}[1]{(\ref{#1})}
\newcommand{\dt}{\mbox{\rm   dt}}
\newcommand{\C}{\mathbb{C}}
\newcommand{\full}{\{1, \ldots, n\}}
\newcommand{\phat}{\hat{p}}
\begin{document}

\section{Hodge results}\label{S:Hodge}

Our goal is to derive an expression for the generating function of the Hodge pieces, that is to prove Theorem 
~\ref{T:Hodge}. We do this by following section 2 of Hanlon \cite{Hanlon}. The idea is to use the following identity
(See proof of Theorem 2.1 in Hanlon \cite{Hanlon}):
\begin{equation} 
  \sum_{j \ge 1} \sum_{p \ge 0} \sum_{\tau \in S_p} [e_p^{(j)}]_\tau Z(\tau) \lambda^j = \prod_l (1 + (-1)^l a_l)^
  {-\frac{1}{l} \sum_{d |l} \mu(d) \lambda^{\frac{l}{d}}}.
\end{equation}



\begin{thebibliography}{9}
\bibitem{GS}
  M. Gerstenhaber and S.D. Schack,
  \emph{A Hodge-type decomposition for commutative algebra cohomology,}
  J. Pure Appl. Algebra {\bf 48} (1987), 229-247.
\bibitem{HanlonMac}
  P. Hanlon,
  \emph{Cyclic homology and the Macdonald conjectures,}
  Invent. Math. {\bf 86} (1986), 131-159.
\bibitem{Hanlon}
  P. Hanlon,
  \emph{Hodge Structure on Posets}
\bibitem{Loday}
  J. L. Loday,
  \emph{Partition eul\'eriene et op\'erations en homologie cyclique.}
  C. R. Acad. Sci. Paris S\'er. I Math. {\bf 307} (1988), 283-286.
\bibitem{Weibel}
  C. Weibel,
  \emph{An introduction to homological algebra,}
  Cambridge University Press, UK, 1994.

\end{thebibliography}
\end{document}
